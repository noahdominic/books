\chapter{Commodity}

\section{The two factors of a commodity:
Usevalue and value
(the substance of value and the magnitude of value)}


The wealth of those societies 
in which the capitalist mode of production prevails
presents itself
as \enquote{an immense accumulation of commodities};\footcite{fn01}
the single commodity appears as the atomic form of wealth.
Our investigation therefore begins
with the analysis of a commodity.

First of all,
a commodity is an external object,
a thing that,
through its characteristics,
satisfies human wants of one sort or another. 
The nature of these wants, 
whether they arise
for example
from the stomach or from fantasy, makes no difference.%
\footnote{\enquote{Desire implies Want; 
it is the Appetite of the Mind,
and as natural as Hunger to the Body.
\textelp{}
The greatest Number of [Things]
have their Value from supplying 
the Wants of the Mind} 
\citep[2--3]{fn02}}
Nor does it matter
how the object satisfies these human wants, 
whether directly as means of subsistence 
or indirectly as means of production.

Every useful thing, such as iron, paper, \&c., 
may be looked at from two points of view: quality and quantity.
Every useful thing is an aggregate of many characteristics
and may therefore be useful in various ways.
The discovery of these characteristics and 
therefore these various ways 
is the work of history.%
\footnote{
    \enquote{Things have a Intrinsick Vertue}
    (this is Barbon's special term for usevalue)
    \enquote{which in all places have the same Vertue;
    as the Loadstone to attract Iron}
    \citep[6]{fn02}
}
So also is the invention of socially-recognised standards of measure
for the quantities of these useful objects.
The diversity of these measures comes
partly from the diverse nature of the objects to be measured
and partly from convention.

The usefulness of a thing makes it a usevalue.%
\footnote{
    \enquote{The natural worth of anything consists 
    in its fitness to supply the necessities
    or serve the conveniencies of human life.}
    \citep[28]{fn04}. 
    In English writers of the 17th century,
    we frequently find \enquote{worth} in the sense of usevalue
    and \enquote{value} in the sense of exchangevalue. 
    This is quite in accordance with the spirit of a language 
    that likes to use a Germanic word for the thing
    and a Latinate for its reflection.
}
But this usefulness does not come out of thin air.
Its existence is dependent on the physical characteristics of the commodity;
it has no existence separate from that of the commodity.
The physical body of the commodity
(such as iron, corn, or diamond)
is therefore a usevalue or a good.
This property of a commodity 
is independent from the amount of labour
required to appropriate its useful qualities.
When examining usevalue,
we always assume to be dealing with definite quantities,
such as dozens of watches,
yards of linen,
or tons of iron.
The usevalues of commodities furnish the material 
for a special branch of knowledge:
merceology.%
\footnote{
    In civil societies,
    the economic \textit{fictio juris} prevails,
    that everyone, as a buyer, possesses an
    encyclopaedic knowledge of commodities
}
Usevalues become a reality only by use or consumption.
They constitute the material substance of wealth,
whatever its social form may be.
In the form of society to be examined here,
they are also the material bearers of exchangevalue.

Exchangevalue, at first sight, presents itself
as a quantitative relation,
the proportion in which usevalues of one kind
are exchanged against the usevalues of another kind,%
\footnote{
    «~La valuer consiste 
    dans le rapport d'échange 
    qui se trouve entre telle chose
    et telle autre,
    entre telle mesure d'une production
    et telle mesure d'une autre.~» 
    (\enquote{Value consists in the exchange relation between
        one thing and another,
        between a mount of one production and an amount of another.})
    \citep[889]{fn06}
}
a relation that constantly changes with time and place.
Hence, exchangevalue appears to be something accidental and purely relative.
An intrinsic exchangevalue 
(i.e.~an exchangevalue that is
inherent with commodity)
therefore seems to be a contradiction in terms.%
\footnote{
    \enquote{Nothing can have an intrinsick Value}~\citep[6]{fn02};
    or as Butler says, 
    \begin{verse}
        \enquote{The value of thing\\
        Is just as much as it will bring.}\\
    \end{verse}
    \citep{butlerhudibras}
}
But let us examine the matter more closely.

A given commodity 
is exchanged for other commodities 
in very different proportions.
For example, 
one quarter of wheat 
is exchanged for 
\(x\) shoe polish,
\(y\) silk,
or \(z\) gold, \&c.;
instead of one exchangevalue,
the wheat has many.
But since 
\(x\) shoe polish,
\(y\) silk,
or \(z\) gold, \&c.,
each represent the exchangevalue of 
one quarter of wheat,
\(x\) shoe polish,
\(y\) silk,
or \(z\) gold, \&c.,
as exchangevalues must also be mutually replaceable,
or equal to each other.
It follows then that
the valid exchangevalues of a given commodity 
signify an equality, and that
exchangevalue itself cannot be anything other than the mode of expression,
the \enquote{form of appearance},
of something contained in a thing yet distinguishable from it.

Let us take two commodities, corn and iron. 
Whatever their exchange relations are,
it can always be represented 
by an equation in which 
a given quantity of corn is equated to some quantity of iron.
For instance, \(1\:\mathrm{kg}_{\mathrm{corn}} = x\:\mathrm{kg}_{\mathrm{iron}}.\)
What does this equation tell us?
That in two different things---in
\(1\:\mathrm{kg}\) of corn and in \(x\:\mathrm{kg}\) of iron, 
there exists, in equal quantities, something common to both.
Both are equal to a third thing, 
which in itself is neither
the one nor the other. 
Each of them, 
so far as it is exchangevalue, 
must therefore be reducible to this third thing.

A simple geometric illustration will make this clear.
In order to calculate and compare
the areas of rectilinear figures,
we decompose them into triangles.
The triangles are then reduced into an expression that is 
different from its visible shape:
\(\frac{hb}{2}\),
half the product of the base times height.
In the same way,
the exchangevalues of commodities 
must be reduced to a common thing, 
of which they represent a greater or lesser quantity.

This common element cannot be either a geometric,
chemical, or any other natural characteristic of commodities.
Those characteristics only become significant to us 
when they affect the usefulness of those commodities,
i.e.~turn them into usevalues.
However, it is exactly this total abstraction 
from usevalue of these commodities 
that characterise this exchange.
One usevalue is just as good as another, 
as long as it is in sufficient quantity.
Or, as old Barbon says, 
\enquote{%
There is no difference nor distinction 
in things of equal Value.
\textelp{}
[O]ne sort of Wares are as good as another,
if the Values be equal. An hundred pounds worth of Lead
or Iron, is as good as an hundred pounds
worth of Silver or Gold.}\footcite[7, 53]{fn02}

As usevalues, 
commodities are, above all, of different qualities,
but as exchangevalues they only differ in quantities
and therefore do not contain an atom of usevalue.

If then we ignore the usevalue of commodities, 
they have only one common characteristic left:
being products of labour.
But even the product of labour itself 
has undergone a change in our hands. 
If we abstract from its usevalue, 
we also abstract from the material elements and shapes 
that make the product a usevalue.
It is no longer a table, a house, yarn, or any other useful thing;
its existence as a material thing 
is erased. 
Nor can it any longer be  
the product of the labour of 
the joiner, the mason, the spinner, or of any other definite kind of productive labour. 
Along with these, 
we put out of sight 
the useful character of the various kinds of labour embodied in them
and the concrete forms of that labour.
There is nothing left but what is common to them all.
All are reduced to the same sort of labour: 
human labour in the abstract.

Let us now consider what remains of each of these products of labour:
in each of them is the same unsubstantial reality, 
a mere congelation of homogeneous human labour, 
i.e.~labourpower spent without regard for how it was spent. 
All that these things now tell us is that 
human labourpower has been expended in their production, 
that human labour is embodied in them. 
As crystals of this social substance, 
which is common to them all, 
they are values---commodity values.

We have seen that when commodities are exchanged, 
their exchangevalue manifests 
as something totally independent of their usevalue. 
But if we abstract from their usevalue, 
there remains their value as defined above. 
Therefore, 
the common factor that emerges 
in the exchangevalue of a commodity, 
whenever it is exchanged, 
is its value. 
The progress of our investigation 
will show that exchangevalue 
is the only form in which 
the value of commodities 
can manifest or be expressed. 
For the present, however, 
we have to consider the nature of value 
independently from its form of appearance.

A usevalue, or a useful thing, 
therefore has value 
only because abstract human labour has been objectified or materialised in it.
How, then, can the quantity of this value be measured?
By the quantity of the value-creating substance,
the labour,
contained in the article.
The quantity of labour, however, is measured by its duration or labourtime,
which finds its standard of measurement 
in well-defined segments of time, like hours, days, \&c.

Some people might think that if the value of a commodity
is determined by its quantity of labour,
then its value would increase
in proportion to the incompetence and laziness of the worker who produced it,
because more labourtime would be needed in its production.
However,
the labour that forms the substance of value
is equal human labour,
expenditure of one uniform labourpower.
The total labourpower of a society,
which is in the sum total of the values 
of all commodities produced in that sociey,
counts here as one uniform mass of human labourpower,
despite being composed of innumerable individual units.
Each units of labourpower
is the same as any other,
to the extent that it has the character of the average labourpower of a society 
and takes effect as such. It therefore requires no more labourtime
than is necessary on an average,
no more than is socially necessary.
The labourtime socially necessary 
is that required to produce an article 
under the normal conditions of production,
and with the average degree of skill and intensity 
prevalent at the time.
The introduction of power looms into England
probably reduced by one-half 
the labour required to weave a given quantity of yarn into cloth.
English hand-loom weavers, as a matter of fact,
continued to require the same time as before,
but after this change,
the product of one hour of their individual labour 
represented only half an hour of social labour
and consequently fell to one-half its former value.

What determines the magnitude of the value of any article 
is therefore only the amount of socially necessary labour,
or the labourtime socially necessary for its production.%
\footnote{
    \enquote{The value of them (the necessaries of life),
    when they are exchanged the one for another,
    is regulated by the quantity of labour
    necessarily required, and commonly taken in producing them.} 
    \citep[36]{fn09}
}
Each individual commodity in this context
count generally as an average sample of its kind.
\footnote{TODO} 
Therefore, commodities that have equal quantities of labour materialised within them
or that can be produced in the same labourtime
have the same magnitudes of value.
The value of one commodity 
is the same as the value of any other,
as the labourtime necessary for the production of one 
is equal to what is necessary for the production of the other.%
\enquote{As values, all commodities are only definite masses
of congealed labourtime.}\footcite{TODO}

The value of a commodity would therefore remain constant,
if the labourtime required for its production also remained constant.
But the latter changes with every variation in the productiveness of labour. 
This productive power is determined 
by various circumstances:
the average skill of workers, 
the state of science and the degree of its practical application,
the social organisation of production processes,
the extent and effectiveness of the means of production,
and the conditions of the natural environemtn, among others.
For example,
the same amount of labour in favourable seasons 
is embodied in 8 bushels of corn,
and in unfavourable, only in four.
The same amount of labour provides more metal in rich mines than in poor ones.
Diamonds are very rare, and so the cost of their discovery 
requires a great deal of labourtime.
Jacob doubts whether gold has ever been paid for at its full value.
This applies even more to diamonds. 
According to Eschwege, the total produce
of the Brazilian diamond mines for the eighty years,
ending in 1823,
sill did not compare to the price of 
one-and-a-half years' average product 
of the sugar and coffee plantations in the same country,
although diamonds represented much more labour
and therefore more value.
With richer mines, 
the same quantity of labour would materialise itself in more diamonds,
and their value would fall. 
If we could succeed at a small expenditure of labour,
in converting carbon into diamonds,
their value might fall below that of bricks.