\chapter{The Little Mermaid}


Far out in the ocean, where the water is as blue as the
prettiest cornflower, and as clear as crystal, it is very, very
deep; so deep, indeed, that no cable could fathom it: many church
steeples, piled one upon another, would not reach from the ground
beneath to the surface of the water above. 
There dwell the Sea King
and his subjects. 
We must not imagine that there is nothing at the
bottom of the sea but bare yellow sand. 
No, indeed; the most
singular flowers and plants grow there; the leaves and stems of
which are so pliant, that the slightest agitation of the water
causes them to stir as if they had life. 
Fishes, both large and small,
glide between the branches, as birds fly among the trees here upon
land. 
In the deepest spot of all, stands the castle of the Sea King.
Its walls are built of coral, and the long, gothic windows are of
the clearest amber. 
The roof is formed of shells, that open and
close as the water flows over them. 
Their appearance is very
beautiful, for in each lies a glittering pearl, which would be fit for
the diadem of a queen.

The Sea King had been a widower for many years, and his aged
mother kept house for him. 
She was a very wise woman, and
exceedingly proud of her high birth; on that account she wore twelve
oysters on her tail; while others, also of high rank, were only
allowed to wear six. 
She was, however, deserving of very great praise,
especially for her care of the little sea-princesses, her
grand-daughters. 
They were six beautiful children; but the youngest
was the prettiest of them all; her skin was as clear and delicate as a
rose-leaf, and her eyes as blue as the deepest sea; but, like all
the others, she had no feet, and her body ended in a fish's tail.
All day long they played in the great halls of the castle, or among
the living flowers that grew out of the walls. 
The large amber windows
were open, and the fish swam in, just as the swallows fly into our
houses when we open the windows, excepting that the fishes swam up
to the princesses, ate out of their hands, and allowed themselves to
be stroked. 
Outside the castle there was a beautiful garden, in
which grew bright red and dark blue flowers, and blossoms like
flames of fire; the fruit glittered like gold, and the leaves and
stems waved to and fro continually. 
The earth itself was the finest
sand, but blue as the flame of burning sulphur. 
Over everything lay
a peculiar blue radiance, as if it were surrounded by the air from
above, through which the blue sky shone, instead of the dark depths of
the sea. 
In calm weather the sun could be seen, looking like a
purple flower, with the light streaming from the calyx. 
Each of the
young princesses had a little plot of ground in the garden, where
she might dig and plant as she pleased. 
One arranged her flower-bed
into the form of a whale; another thought it better to make hers
like the figure of a little mermaid; but that of the youngest was
round like the sun, and contained flowers as red as his rays at
sunset. 
She was a strange child, quiet and thoughtful; and while her
sisters would be delighted with the wonderful things which they
obtained from the wrecks of vessels, she cared for nothing but her
pretty red flowers, like the sun, excepting a beautiful marble statue.
It was the representation of a handsome boy, carved out of pure
white stone, which had fallen to the bottom of the sea from a wreck.
She planted by the statue a rose-colored weeping willow. 
It grew
splendidly, and very soon hung its fresh branches over the statue,
almost down to the blue sands. 
The shadow had a violet tint, and waved
to and fro like the branches; it seemed as if the crown of the tree
and the root were at play, and trying to kiss each other. 
Nothing gave
her so much pleasure as to hear about the world above the sea. 
She
made her old grandmother tell her all she knew of the ships and of the
towns, the people and the animals. 
To her it seemed most wonderful and
beautiful to hear that the flowers of the land should have
fragrance, and not those below the sea; that the trees of the forest
should be green; and that the fishes among the trees could sing so
sweetly, that it was quite a pleasure to hear them. 
Her grandmother
called the little birds fishes, or she would not have understood
her; for she had never seen birds.

`When you have reached your fifteenth year,' said the
grand-mother, `you will have permission to rise up out of the sea,
to sit on the rocks in the moonlight, while the great ships are
sailing by; and then you will see both forests and towns.'

In the following year, one of the sisters would be fifteen: but as
each was a year younger than the other, the youngest would have to
wait five years before her turn came to rise up from the bottom of the
ocean, and see the earth as we do. 
However, each promised to tell
the others what she saw on her first visit, and what she thought the
most beautiful; for their grandmother could not tell them enough;
there were so many things on which they wanted information. 
None of
them longed so much for her turn to come as the youngest, she who
had the longest time to wait, and who was so quiet and thoughtful.
Many nights she stood by the open window, looking up through the
dark blue water, and watching the fish as they splashed about with
their fins and tails. 
She could see the moon and stars shining
faintly; but through the water they looked larger than they do to
our eyes. 
When something like a black cloud passed between her and
them, she knew that it was either a whale swimming over her head, or a
ship full of human beings, who never imagined that a pretty little
mermaid was standing beneath them, holding out her white hands towards
the keel of their ship.

As soon as the eldest was fifteen, she was allowed to rise to
the surface of the ocean. 
When she came back, she had hundreds of
things to talk about; but the most beautiful, she said, was to lie
in the moonlight, on a sandbank, in the quiet sea, near the coast, and
to gaze on a large town nearby, where the lights were twinkling like
hundreds of stars; to listen to the sounds of the music, the noise
of carriages, and the voices of human beings, and then to hear the
merry bells peal out from the church steeples; and because she could
not go near to all those wonderful things, she longed for them more
than ever. 
Oh, did not the youngest sister listen eagerly to all these
descriptions? and afterwards, when she stood at the open window
looking up through the dark blue water, she thought of the great city,
with all its bustle and noise, and even fancied she could hear the
sound of the church bells, down in the depths of the sea.

In another year the second sister received permission to rise to
the surface of the water, and to swim about where she pleased. 
She
rose just as the sun was setting, and this, she said, was the most
beautiful sight of all. 
The whole sky looked like gold, while violet
and rose-colored clouds, which she could not describe, floated over
her; and, still more rapidly than the clouds, flew a large flock of
wild swans towards the setting sun, looking like a long white veil
across the sea. 
She also swam towards the sun; but it sunk into the
waves, and the rosy tints faded from the clouds and from the sea.

The third sister's turn followed; she was the boldest of them all,
and she swam up a broad river that emptied itself into the sea. 
On the
banks she saw green hills covered with beautiful vines; palaces and
castles peeped out from amid the proud trees of the forest; she
heard the birds singing, and the rays of the sun were so powerful that
she was obliged often to dive down under the water to cool her burning
face. 
In a narrow creek she found a whole troop of little human
children, quite naked, and sporting about in the water; she wanted
to play with them, but they fled in a great fright; and then a
little black animal came to the water; it was a dog, but she did not
know that, for she had never before seen one. 
This animal barked at
her so terribly that she became frightened, and rushed back to the
open sea. 
But she said she should never forget the beautiful forest,
the green hills, and the pretty little children who could swim in
the water, although they had not fish's tails.

The fourth sister was more timid; she remained in the midst of the
sea, but she said it was quite as beautiful there as nearer the
land. 
She could see for so many miles around her, and the sky above
looked like a bell of glass. 
She had seen the ships, but at such a
great distance that they looked like sea-gulls. 
The dolphins sported
in the waves, and the great whales spouted water from their nostrils
till it seemed as if a hundred fountains were playing in every
direction.

The fifth sister's birthday occurred in the winter; so when her
turn came, she saw what the others had not seen the first time they
went up. 
The sea looked quite green, and large icebergs were
floating about, each like a pearl, she said, but larger and loftier
than the churches built by men. 
They were of the most singular shapes,
and glittered like diamonds. 
She had seated herself upon one of the
largest, and let the wind play with her long hair, and she remarked
that all the ships sailed by rapidly, and steered as far away as
they could from the iceberg, as if they were afraid of it. 
Towards
evening, as the sun went down, dark clouds covered the sky, the
thunder rolled and the lightning flashed, and the red light glowed
on the icebergs as they rocked and tossed on the heaving sea. 
On all
the ships the sails were reefed with fear and trembling, while she sat
calmly on the floating iceberg, watching the blue lightning, as it
darted its forked flashes into the sea.

When first the sisters had permission to rise to the surface, they
were each delighted with the new and beautiful sights they saw; but
now, as grown-up girls, they could go when they pleased, and they
had become indifferent about it. 
They wished themselves back again
in the water, and after a month had passed they said it was much
more beautiful down below, and pleasanter to be at home. 
Yet often, in
the evening hours, the five sisters would twine their arms round
each other, and rise to the surface, in a row. 
They had more beautiful
voices than any human being could have; and before the approach of a
storm, and when they expected a ship would be lost, they swam before
the vessel, and sang sweetly of the delights to be found in the depths
of the sea, and begging the sailors not to fear if they sank to the
bottom. 
But the sailors could not understand the song, they took it
for the howling of the storm. 
And these things were never to be
beautiful for them; for if the ship sank, the men were drowned, and
their dead bodies alone reached the palace of the Sea King.

When the sisters rose, arm-in-arm, through the water in this
way, their youngest sister would stand quite alone, looking after
them, ready to cry, only that the mermaids have no tears, and
therefore they suffer more. 
`Oh, were I but fifteen years old,' said
she: `I know that I shall love the world up there, and all the
people who live in it.'

At last she reached her fifteenth year. 
`Well, now, you are
grown up,' said the old dowager, her grandmother; `so you must let
me adorn you like your other sisters;' and she placed a wreath of
white lilies in her hair, and every flower leaf was half a pearl. 
Then
the old lady ordered eight great oysters to attach themselves to the
tail of the princess to show her high rank.

`But they hurt me so,' said the little mermaid.

`Pride must suffer pain,' replied the old lady. 
Oh, how gladly she
would have shaken off all this grandeur, and laid aside the heavy
wreath! The red flowers in her own garden would have suited her much
better, but she could not help herself: so she said, `Farewell,' and
rose as lightly as a bubble to the surface of the water. 
The sun had
just set as she raised her head above the waves; but the clouds were
tinted with crimson and gold, and through the glimmering twilight
beamed the evening star in all its beauty. 
The sea was calm, and the
air mild and fresh. 
A large ship, with three masts, lay becalmed on
the water, with only one sail set; for not a breeze stiffed, and the
sailors sat idle on deck or amongst the rigging. 
There was music and
song on board; and, as darkness came on, a hundred colored lanterns
were lighted, as if the flags of all nations waved in the air. 
The
little mermaid swam close to the cabin windows; and now and then, as
the waves lifted her up, she could look in through clear glass
window-panes, and see a number of well-dressed people within. 
Among
them was a young prince, the most beautiful of all, with large black
eyes; he was sixteen years of age, and his birthday was being kept
with much rejoicing. 
The sailors were dancing on deck, but when the
prince came out of the cabin, more than a hundred rockets rose in
the air, making it as bright as day. 
The little mermaid was so
startled that she dived under water; and when she again stretched
out her head, it appeared as if all the stars of heaven were falling
around her, she had never seen such fireworks before. 
Great suns
spurted fire about, splendid fireflies flew into the blue air, and
everything was reflected in the clear, calm sea beneath. 
The ship
itself was so brightly illuminated that all the people, and even the
smallest rope, could be distinctly and plainly seen. 
And how
handsome the young prince looked, as he pressed the hands of all
present and smiled at them, while the music resounded through the
clear night air.

It was very late; yet the little mermaid could not take her eyes
from the ship, or from the beautiful prince. 
The colored lanterns
had been extinguished, no more rockets rose in the air, and the cannon
had ceased firing; but the sea became restless, and a moaning,
grumbling sound could be heard beneath the waves: still the little
mermaid remained by the cabin window, rocking up and down on the
water, which enabled her to look in. 
After a while, the sails were
quickly unfurled, and the noble ship continued her passage; but soon
the waves rose higher, heavy clouds darkened the sky, and lightning
appeared in the distance. 
A dreadful storm was approaching; once
more the sails were reefed, and the great ship pursued her flying
course over the raging sea. 
The waves rose mountains high, as if
they would have overtopped the mast; but the ship dived like a swan
between them, and then rose again on their lofty, foaming crests. 
To
the little mermaid this appeared pleasant sport; not so to the
sailors. 
At length the ship groaned and creaked; the thick planks gave
way under the lashing of the sea as it broke over the deck; the
mainmast snapped asunder like a reed; the ship lay over on her side;
and the water rushed in. 
The little mermaid now perceived that the
crew were in danger; even she herself was obliged to be careful to
avoid the beams and planks of the wreck which lay scattered on the
water. 
At one moment it was so pitch dark that she could not see a
single object, but a flash of lightning revealed the whole scene;
she could see every one who had been on board excepting the prince;
when the ship parted, she had seen him sink into the deep waves, and
she was glad, for she thought he would now be with her; and then she
remembered that human beings could not live in the water, so that when
he got down to her father's palace he would be quite dead. 
But he must
not die. 
So she swam about among the beams and planks which strewed
the surface of the sea, forgetting that they could crush her to
pieces. 
Then she dived deeply under the dark waters, rising and
falling with the waves, till at length she managed to reach the
young prince, who was fast losing the power of swimming in that stormy
sea. 
His limbs were failing him, his beautiful eyes were closed, and
he would have died had not the little mermaid come to his
assistance. 
She held his head above the water, and let the waves drift
them where they would.

In the morning the storm had ceased; but of the ship not a
single fragment could be seen. 
The sun rose up red and glowing from
the water, and its beams brought back the hue of health to the
prince's cheeks; but his eyes remained closed. 
The mermaid kissed
his high, smooth forehead, and stroked back his wet hair; he seemed to
her like the marble statue in her little garden, and she kissed him
again, and wished that he might live. 
Presently they came in sight
of land; she saw lofty blue mountains, on which the white snow
rested as if a flock of swans were lying upon them. 
Near the coast
were beautiful green forests, and close by stood a large building,
whether a church or a convent she could not tell. 
Orange and citron
trees grew in the garden, and before the door stood lofty palms. 
The
sea here formed a little bay, in which the water was quite still,
but very deep; so she swam with the handsome prince to the beach,
which was covered with fine, white sand, and there she laid him in the
warm sunshine, taking care to raise his head higher than his body.
Then bells sounded in the large white building, and a number of
young girls came into the garden. 
The little mermaid swam out
farther from the shore and placed herself between some high rocks that
rose out of the water; then she covered her head and neck with the
foam of the sea so that her little face might not be seen, and watched
to see what would become of the poor prince. 
She did not wait long
before she saw a young girl approach the spot where he lay. 
She seemed
frightened at first, but only for a moment; then she fetched a
number of people, and the mermaid saw that the prince came to life
again, and smiled upon those who stood round him. 
But to her he sent
no smile; he knew not that she had saved him. 
This made her very
unhappy, and when he was led away into the great building, she dived
down sorrowfully into the water, and returned to her father's
castle. 
She had always been silent and thoughtful, and now she was
more so than ever. 
Her sisters asked her what she had seen during
her first visit to the surface of the water; but she would tell them
nothing. 
Many an evening and morning did she rise to the place where
she had left the prince. 
She saw the fruits in the garden ripen till
they were gathered, the snow on the tops of the mountains melt away;
but she never saw the prince, and therefore she returned home,
always more sorrowful than before. 
It was her only comfort to sit in
her own little garden, and fling her arm round the beautiful marble
statue which was like the prince; but she gave up tending her flowers,
and they grew in wild confusion over the paths, twining their long
leaves and stems round the branches of the trees, so that the whole
place became dark and gloomy. 
At length she could bear it no longer,
and told one of her sisters all about it. 
Then the others heard the
secret, and very soon it became known to two mermaids whose intimate
friend happened to know who the prince was. 
She had also seen the
festival on board ship, and she told them where the prince came
from, and where his palace stood.

`Come, little sister,' said the other princesses; then they
entwined their arms and rose up in a long row to the surface of the
water, close by the spot where they knew the prince's palace stood. 
It
was built of bright yellow shining stone, with long flights of
marble steps, one of which reached quite down to the sea. 
Splendid
gilded cupolas rose over the roof, and between the pillars that
surrounded the whole building stood life-like statues of marble.
Through the clear crystal of the lofty windows could be seen noble
rooms, with costly silk curtains and hangings of tapestry; while the
walls were covered with beautiful paintings which were a pleasure to
look at. 
In the centre of the largest saloon a fountain threw its
sparkling jets high up into the glass cupola of the ceiling, through
which the sun shone down upon the water and upon the beautiful
plants growing round the basin of the fountain. 
Now that she knew
where he lived, she spent many an evening and many a night on the
water near the palace. 
She would swim much nearer the shore than any
of the others ventured to do; indeed once she went quite up the narrow
channel under the marble balcony, which threw a broad shadow on the
water. 
Here she would sit and watch the young prince, who thought
himself quite alone in the bright moonlight. 
She saw him many times of
an evening sailing in a pleasant boat, with music playing and flags
waving. 
She peeped out from among the green rushes, and if the wind
caught her long silvery-white veil, those who saw it believed it to be
a swan, spreading out its wings. 
On many a night, too, when the
fishermen, with their torches, were out at sea, she heard them
relate so many good things about the doings of the young prince,
that she was glad she had saved his life when he had been tossed about
half-dead on the waves. 
And she remembered that his head had rested on
her bosom, and how heartily she had kissed him; but he knew nothing of
all this, and could not even dream of her. 
She grew more and more fond
of human beings, and wished more and more to be able to wander about
with those whose world seemed to be so much larger than her own.
They could fly over the sea in ships, and mount the high hills which
were far above the clouds; and the lands they possessed, their woods
and their fields, stretched far away beyond the reach of her sight.
There was so much that she wished to know, and her sisters were unable
to answer all her questions. 
Then she applied to her old
grandmother, who knew all about the upper world, which she very
rightly called the lands above the sea.

`If human beings are not drowned,' asked the little mermaid,
`can they live forever? do they never die as we do here in the sea?'

`Yes,' replied the old lady, `they must also die, and their term
of life is even shorter than ours. 
We sometimes live to three
hundred years, but when we cease to exist here we only become the foam
on the surface of the water, and we have not even a grave down here of
those we love. 
We have not immortal souls, we shall never live
again; but, like the green sea-weed, when once it has been cut off, we
can never flourish more. 
Human beings, on the contrary, have a soul
which lives forever, lives after the body has been turned to dust.
It rises up through the clear, pure air beyond the glittering stars.
As we rise out of the water, and behold all the land of the earth,
so do they rise to unknown and glorious regions which we shall never
see.'

`Why have not we an immortal soul?' asked the little mermaid
mournfully; `I would give gladly all the hundreds of years that I have
to live, to be a human being only for one day, and to have the hope of
knowing the happiness of that glorious world above the stars.'

`You must not think of that,' said the old woman; `we feel
ourselves to be much happier and much better off than human beings.'

`So I shall die,' said the little mermaid, `and as the foam of the
sea I shall be driven about never again to hear the music of the
waves, or to see the pretty flowers nor the red sun. 
Is there anything
I can do to win an immortal soul?'

`No,' said the old woman, `unless a man were to love you so much
that you were more to him than his father or mother; and if all his
thoughts and all his love were fixed upon you, and the priest placed
his right hand in yours, and he promised to be true to you here and
hereafter, then his soul would glide into your body and you would
obtain a share in the future happiness of mankind. 
He would give a
soul to you and retain his own as well; but this can never happen.
Your fish's tail, which amongst us is considered so beautiful, is
thought on earth to be quite ugly; they do not know any better, and
they think it necessary to have two stout props, which they call legs,
in order to be handsome.'

Then the little mermaid sighed, and looked sorrowfully at her
fish's tail. 
`Let us be happy,' said the old lady, `and dart and
spring about during the three hundred years that we have to live,
which is really quite long enough; after that we can rest ourselves
all the better. 
This evening we are going to have a court ball.'

It is one of those splendid sights which we can never see on
earth. 
The walls and the ceiling of the large ball-room were of thick,
but transparent crystal. 
May hundreds of colossal shells, some of a
deep red, others of a grass green, stood on each side in rows, with
blue fire in them, which lighted up the whole saloon, and shone
through the walls, so that the sea was also illuminated. 
Innumerable
fishes, great and small, swam past the crystal walls; on some of
them the scales glowed with a purple brilliancy, and on others they
shone like silver and gold. 
Through the halls flowed a broad stream,
and in it danced the mermen and the mermaids to the music of their own
sweet singing. 
No one on earth has such a lovely voice as theirs.
The little mermaid sang more sweetly than them all. 
The whole court
applauded her with hands and tails; and for a moment her heart felt
quite gay, for she knew she had the loveliest voice of any on earth or
in the sea. 
But she soon thought again of the world above her, for she
could not forget the charming prince, nor her sorrow that she had
not an immortal soul like his; therefore she crept away silently out
of her father's palace, and while everything within was gladness and
song, she sat in her own little garden sorrowful and alone. 
Then she
heard the bugle sounding through the water, and thought- `He is
certainly sailing above, he on whom my wishes depend, and in whose
hands I should like to place the happiness of my life. 
I will
venture all for him, and to win an immortal soul, while my sisters are
dancing in my father's palace, I will go to the sea witch, of whom I
have always been so much afraid, but she can give me counsel and
help.'

And then the little mermaid went out from her garden, and took the
road to the foaming whirlpools, behind which the sorceress lived.
She had never been that way before: neither flowers nor grass grew
there; nothing but bare, gray, sandy ground stretched out to the
whirlpool, where the water, like foaming mill-wheels, whirled round
everything that it seized, and cast it into the fathomless deep.
Through the midst of these crushing whirlpools the little mermaid
was obliged to pass, to reach the dominions of the sea witch; and also
for a long distance the only road lay right across a quantity of warm,
bubbling mire, called by the witch her turfmoor. 
Beyond this stood her
house, in the centre of a strange forest, in which all the trees and
flowers were polypi, half animals and half plants; they looked like
serpents with a hundred heads growing out of the ground. 
The
branches were long slimy arms, with fingers like flexible worms,
moving limb after limb from the root to the top. 
All that could be
reached in the sea they seized upon, and held fast, so that it never
escaped from their clutches. 
The little mermaid was so alarmed at what
she saw, that she stood still, and her heart beat with fear, and she
was very nearly turning back; but she thought of the prince, and of
the human soul for which she longed, and her courage returned. 
She
fastened her long flowing hair round her head, so that the polypi
might not seize hold of it. 
She laid her hands together across her
bosom, and then she darted forward as a fish shoots through the water,
between the supple arms and fingers of the ugly polypi, which were
stretched out on each side of her. 
She saw that each held in its grasp
something it had seized with its numerous little arms, as if they were
iron bands. 
The white skeletons of human beings who had perished at
sea, and had sunk down into the deep waters, skeletons of land
animals, oars, rudders, and chests of ships were lying tightly grasped
by their clinging arms; even a little mermaid, whom they had caught
and strangled; and this seemed the most shocking of all to the
little princess.

She now came to a space of marshy ground in the wood, where large,
fat water-snakes were rolling in the mire, and showing their ugly,
drab-colored bodies. 
In the midst of this spot stood a house, built
with the bones of shipwrecked human beings. 
There sat the sea witch,
allowing a toad to eat from her mouth, just as people sometimes feed a
canary with a piece of sugar. 
She called the ugly water-snakes her
little chickens, and allowed them to crawl all over her bosom.

`I know what you want,' said the sea witch; `it is very stupid
of you, but you shall have your way, and it will bring you to
sorrow, my pretty princess. 
You want to get rid of your fish's tail,
and to have two supports instead of it, like human beings on earth, so
that the young prince may fall in love with you, and that you may have
an immortal soul.' And then the witch laughed so loud and
disgustingly, that the toad and the snakes fell to the ground, and lay
there wriggling about. 
`You are but just in time,' said the witch;
`for after sunrise to-morrow I should not be able to help you till the
end of another year. 
I will prepare a draught for you, with which
you must swim to land tomorrow before sunrise, and sit down on the
shore and drink it. 
Your tail will then disappear, and shrink up
into what mankind calls legs, and you will feel great pain, as if a
sword were passing through you. 
But all who see you will say that
you are the prettiest little human being they ever saw. 
You will still
have the same floating gracefulness of movement, and no dancer will
ever tread so lightly; but at every step you take it will feel as if
you were treading upon sharp knives, and that the blood must flow.
If you will bear all this, I will help you.'

`Yes, I will,' said the little princess in a trembling voice, as
she thought of the prince and the immortal soul.

`But think again,' said the witch; `for when once your shape has
become like a human being, you can no more be a mermaid. 
You will
never return through the water to your sisters, or to your father's
palace again; and if you do not win the love of the prince, so that he
is willing to forget his father and mother for your sake, and to
love you with his whole soul, and allow the priest to join your
hands that you may be man and wife, then you will never have an
immortal soul. 
The first morning after he marries another your heart
will break, and you will become foam on the crest of the waves.'

`I will do it,' said the little mermaid, and she became pale as
death.

`But I must be paid also,' said the witch, `and it is not a trifle
that I ask. 
You have the sweetest voice of any who dwell here in the
depths of the sea, and you believe that you will be able to charm
the prince with it also, but this voice you must give to me; the
best thing you possess will I have for the price of my draught. 
My own
blood must be mixed with it, that it may be as sharp as a two-edged
sword.'

`But if you take away my voice,' said the little mermaid, `what is
left for me?'

`Your beautiful form, your graceful walk, and your expressive
eyes; surely with these you can enchain a man's heart. 
Well, have
you lost your courage? Put out your little tongue that I may cut it
off as my payment; then you shall have the powerful draught.'

`It shall be,' said the little mermaid.

Then the witch placed her cauldron on the fire, to prepare the
magic draught.

`Cleanliness is a good thing,' said she, scouring the vessel
with snakes, which she had tied together in a large knot; then she
pricked herself in the breast, and let the black blood drop into it.
The steam that rose formed itself into such horrible shapes that no
one could look at them without fear. 
Every moment the witch threw
something else into the vessel, and when it began to boil, the sound
was like the weeping of a crocodile. 
When at last the magic draught
was ready, it looked like the clearest water. 
`There it is for you,'
said the witch. 
Then she cut off the mermaid's tongue, so that she
became dumb, and would never again speak or sing. 
`If the polypi
should seize hold of you as you return through the wood,' said the
witch, `throw over them a few drops of the potion, and their fingers
will be torn into a thousand pieces.' But the little mermaid had no
occasion to do this, for the polypi sprang back in terror when they
caught sight of the glittering draught, which shone in her hand like a
twinkling star.

So she passed quickly through the wood and the marsh, and
between the rushing whirlpools. 
She saw that in her father's palace
the torches in the ballroom were extinguished, and all within
asleep; but she did not venture to go in to them, for now she was dumb
and going to leave them forever, she felt as if her heart would break.
She stole into the garden, took a flower from the flower-beds of
each of her sisters, kissed her hand a thousand times towards the
palace, and then rose up through the dark blue waters. 
The sun had not
risen when she came in sight of the prince's palace, and approached
the beautiful marble steps, but the moon shone clear and bright.
Then the little mermaid drank the magic draught, and it seemed as if a
two-edged sword went through her delicate body: she fell into a swoon,
and lay like one dead. 
When the sun arose and shone over the sea,
she recovered, and felt a sharp pain; but just before her stood the
handsome young prince. 
He fixed his coal-black eyes upon her so
earnestly that she cast down her own, and then became aware that her
fish's tail was gone, and that she had as pretty a pair of white
legs and tiny feet as any little maiden could have; but she had no
clothes, so she wrapped herself in her long, thick hair. 
The prince
asked her who she was, and where she came from, and she looked at
him mildly and sorrowfully with her deep blue eyes; but she could
not speak. 
Every step she took was as the witch had said it would
be, she felt as if treading upon the points of needles or sharp
knives; but she bore it willingly, and stepped as lightly by the
prince's side as a soap-bubble, so that he and all who saw her
wondered at her graceful-swaying movements. 
She was very soon
arrayed in costly robes of silk and muslin, and was the most beautiful
creature in the palace; but she was dumb, and could neither speak
nor sing.

Beautiful female slaves, dressed in silk and gold, stepped forward
and sang before the prince and his royal parents: one sang better than
all the others, and the prince clapped his hands and smiled at her.
This was great sorrow to the little mermaid; she knew how much more
sweetly she herself could sing once, and she thought, `Oh if he
could only know that! I have given away my voice forever, to be with
him.'

The slaves next performed some pretty fairy-like dances, to the
sound of beautiful music. 
Then the little mermaid raised her lovely
white arms, stood on the tips of her toes, and glided over the
floor, and danced as no one yet had been able to dance. 
At each moment
her beauty became more revealed, and her expressive eyes appealed more
directly to the heart than the songs of the slaves. 
Every one was
enchanted, especially the prince, who called her his little foundling;
and she danced again quite readily, to please him, though each time
her foot touched the floor it seemed as if she trod on sharp knives.'

The prince said she should remain with him always, and she
received permission to sleep at his door, on a velvet cushion. 
He
had a page's dress made for her, that she might accompany him on
horseback. 
They rode together through the sweet-scented woods, where
the green boughs touched their shoulders, and the little birds sang
among the fresh leaves. 
She climbed with the prince to the tops of
high mountains; and although her tender feet bled so that even her
steps were marked, she only laughed, and followed him till they
could see the clouds beneath them looking like a flock of birds
travelling to distant lands. 
While at the prince's palace, and when
all the household were asleep, she would go and sit on the broad
marble steps; for it eased her burning feet to bathe them in the
cold sea-water; and then she thought of all those below in the deep.

Once during the night her sisters came up arm-in-arm, singing
sorrowfully, as they floated on the water. 
She beckoned to them, and
then they recognized her, and told her how she had grieved them. 
After
that, they came to the same place every night; and once she saw in the
distance her old grandmother, who had not been to the surface of the
sea for many years, and the old Sea King, her father, with his crown
on his head. 
They stretched out their hands towards her, but they
did not venture so near the land as her sisters did.

As the days passed, she loved the prince more fondly, and he loved
her as he would love a little child, but it never came into his head
to make her his wife; yet, unless he married her, she could not
receive an immortal soul; and, on the morning after his marriage
with another, she would dissolve into the foam of the sea.

`Do you not love me the best of them all?' the eyes of the
little mermaid seemed to say, when he took her in his arms, and kissed
her fair forehead.

`Yes, you are dear to me,' said the prince; `for you have the best
heart, and you are the most devoted to me; you are like a young maiden
whom I once saw, but whom I shall never meet again. 
I was in a ship
that was wrecked, and the waves cast me ashore near a holy temple,
where several young maidens performed the service. 
The youngest of
them found me on the shore, and saved my life. 
I saw her but twice,
and she is the only one in the world whom I could love; but you are
like her, and you have almost driven her image out of my mind. 
She
belongs to the holy temple, and my good fortune has sent you to me
instead of her; and we will never part.'

`Ah, he knows not that it was I who saved his life,' thought the
little mermaid. 
`I carried him over the sea to the wood where the
temple stands: I sat beneath the foam, and watched till the human
beings came to help him. 
I saw the pretty maiden that he loves
better than he loves me;' and the mermaid sighed deeply, but she could
not shed tears. 
`He says the maiden belongs to the holy temple,
therefore she will never return to the world. 
They will meet no
more: while I am by his side, and see him every day. 
I will take
care of him, and love him, and give up my life for his sake.'

Very soon it was said that the prince must marry, and that the
beautiful daughter of a neighboring king would be his wife, for a fine
ship was being fitted out. 
Although the prince gave out that he merely
intended to pay a visit to the king, it was generally supposed that he
really went to see his daughter. 
A great company were to go with
him. 
The little mermaid smiled, and shook her head. 
She knew the
prince's thoughts better than any of the others.

`I must travel,' he had said to her; `I must see this beautiful
princess; my parents desire it; but they will not oblige me to bring
her home as my bride. 
I cannot love her; she is not like the beautiful
maiden in the temple, whom you resemble. 
If I were forced to choose
a bride, I would rather choose you, my dumb foundling, with those
expressive eyes.' And then he kissed her rosy mouth, played with her
long waving hair, and laid his head on her heart, while she dreamed of
human happiness and an immortal soul. 
`You are not afraid of the
sea, my dumb child,' said he, as they stood on the deck of the noble
ship which was to carry them to the country of the neighboring king.
And then he told her of storm and of calm, of strange fishes in the
deep beneath them, and of what the divers had seen there; and she
smiled at his descriptions, for she knew better than any one what
wonders were at the bottom of the sea.

In the moonlight, when all on board were asleep, excepting the man
at the helm, who was steering, she sat on the deck, gazing down
through the clear water. 
She thought she could distinguish her
father's castle, and upon it her aged grandmother, with the silver
crown on her head, looking through the rushing tide at the keel of the
vessel. 
Then her sisters came up on the waves, and gazed at her
mournfully, wringing their white hands. 
She beckoned to them, and
smiled, and wanted to tell them how happy and well off she was; but
the cabin-boy approached, and when her sisters dived down he thought
it was only the foam of the sea which he saw.

The next morning the ship sailed into the harbor of a beautiful
town belonging to the king whom the prince was going to visit. 
The
church bells were ringing, and from the high towers sounded a flourish
of trumpets; and soldiers, with flying colors and glittering bayonets,
lined the rocks through which they passed. 
Every day was a festival;
balls and entertainments followed one another.

But the princess had not yet appeared. 
People said that she was
being brought up and educated in a religious house, where she was
learning every royal virtue. 
At last she came. 
Then the little
mermaid, who was very anxious to see whether she was really beautiful,
was obliged to acknowledge that she had never seen a more perfect
vision of beauty. 
Her skin was delicately fair, and beneath her long
dark eye-lashes her laughing blue eyes shone with truth and purity.

`It was you,' said the prince, `who saved my life when I lay
dead on the beach,' and he folded his blushing bride in his arms. 
`Oh,
I am too happy,' said he to the little mermaid; `my fondest hopes
are all fulfilled. 
You will rejoice at my happiness; for your devotion
to me is great and sincere.'

The little mermaid kissed his hand, and felt as if her heart
were already broken. 
His wedding morning would bring death to her, and
she would change into the foam of the sea. 
All the church bells
rung, and the heralds rode about the town proclaiming the betrothal.
Perfumed oil was burning in costly silver lamps on every altar. 
The
priests waved the censers, while the bride and bridegroom joined their
hands and received the blessing of the bishop. 
The little mermaid,
dressed in silk and gold, held up the bride's train; but her ears
heard nothing of the festive music, and her eyes saw not the holy
ceremony; she thought of the night of death which was coming to her,
and of all she had lost in the world. 
On the same evening the bride
and bridegroom went on board ship; cannons were roaring, flags waving,
and in the centre of the ship a costly tent of purple and gold had
been erected. 
It contained elegant couches, for the reception of the
bridal pair during the night. 
The ship, with swelling sails and a
favorable wind, glided away smoothly and lightly over the calm sea.
When it grew dark a number of colored lamps were lit, and the
sailors danced merrily on the deck. 
The little mermaid could not
help thinking of her first rising out of the sea, when she had seen
similar festivities and joys; and she joined in the dance, poised
herself in the air as a swallow when he pursues his prey, and all
present cheered her with wonder. 
She had never danced so elegantly
before. 
Her tender feet felt as if cut with sharp knives, but she
cared not for it; a sharper pang had pierced through her heart. 
She
knew this was the last evening she should ever see the prince, for
whom she had forsaken her kindred and her home; she had given up her
beautiful voice, and suffered unheard-of pain daily for him, while
he knew nothing of it. 
This was the last evening that she would
breathe the same air with him, or gaze on the starry sky and the
deep sea; an eternal night, without a thought or a dream, awaited her:
she had no soul and now she could never win one. 
All was joy and
gayety on board ship till long after midnight; she laughed and
danced with the rest, while the thoughts of death were in her heart.
The prince kissed his beautiful bride, while she played with his raven
hair, till they went arm-in-arm to rest in the splendid tent. 
Then all
became still on board the ship; the helmsman, alone awake, stood at
the helm. 
The little mermaid leaned her white arms on the edge of
the vessel, and looked towards the east for the first blush of
morning, for that first ray of dawn that would bring her death. 
She
saw her sisters rising out of the flood: they were as pale as herself;
but their long beautiful hair waved no more in the wind, and had
been cut off.

`We have given our hair to the witch,' said they, `to obtain
help for you, that you may not die to-night. 
She has given us a knife:
here it is, see it is very sharp. 
Before the sun rises you must plunge
it into the heart of the prince; when the warm blood falls upon your
feet they will grow together again, and form into a fish's tail, and
you will be once more a mermaid, and return to us to live out your
three hundred years before you die and change into the salt sea
foam. 
Haste, then; he or you must die before sunrise. 
Our old
grandmother moans so for you, that her white hair is falling off
from sorrow, as ours fell under the witch's scissors. 
Kill the
prince and come back; hasten: do you not see the first red streaks
in the sky? In a few minutes the sun will rise, and you must die.' And
then they sighed deeply and mournfully, and sank down beneath the
waves.

The little mermaid drew back the crimson curtain of the tent,
and beheld the fair bride with her head resting on the prince's
breast. 
She bent down and kissed his fair brow, then looked at the sky
on which the rosy dawn grew brighter and brighter; then she glanced at
the sharp knife, and again fixed her eyes on the prince, who whispered
the name of his bride in his dreams. 
She was in his thoughts, and
the knife trembled in the hand of the little mermaid: then she flung
it far away from her into the waves; the water turned red where it
fell, and the drops that spurted up looked like blood. 
She cast one
more lingering, half-fainting glance at the prince, and then threw
herself from the ship into the sea, and thought her body was
dissolving into foam. 
The sun rose above the waves, and his warm
rays fell on the cold foam of the little mermaid, who did not feel
as if she were dying. 
She saw the bright sun, and all around her
floated hundreds of transparent beautiful beings; she could see
through them the white sails of the ship, and the red clouds in the
sky; their speech was melodious, but too ethereal to be heard by
mortal ears, as they were also unseen by mortal eyes. 
The little
mermaid perceived that she had a body like theirs, and that she
continued to rise higher and higher out of the foam. 
`Where am I?'
asked she, and her voice sounded ethereal, as the voice of those who
were with her; no earthly music could imitate it.

`Among the daughters of the air,' answered one of them. 
`A mermaid
has not an immortal soul, nor can she obtain one unless she wins the
love of a human being. 
On the power of another hangs her eternal
destiny. 
But the daughters of the air, although they do not possess an
immortal soul, can, by their good deeds, procure one for themselves.
We fly to warm countries, and cool the sultry air that destroys
mankind with the pestilence. 
We carry the perfume of the flowers to
spread health and restoration. 
After we have striven for three hundred
years to all the good in our power, we receive an immortal soul and
take part in the happiness of mankind. 
You, poor little mermaid,
have tried with your whole heart to do as we are doing; you have
suffered and endured and raised yourself to the spirit-world by your
good deeds; and now, by striving for three hundred years in the same
way, you may obtain an immortal soul.'

The little mermaid lifted her glorified eyes towards the sun,
and felt them, for the first time, filling with tears. 
On the ship, in
which she had left the prince, there were life and noise; she saw
him and his beautiful bride searching for her; sorrowfully they
gazed at the pearly foam, as if they knew she had thrown herself
into the waves. 
Unseen she kissed the forehead of her bride, and
fanned the prince, and then mounted with the other children of the air
to a rosy cloud that floated through the aether.

`After three hundred years, thus shall we float into the kingdom
of heaven,' said she. 
`And we may even get there sooner,' whispered
one of her companions. 
`Unseen we can enter the houses of men, where
there are children, and for every day on which we find a good child,
who is the joy of his parents and deserves their love, our time of
probation is shortened. 
The child does not know, when we fly through
the room, that we smile with joy at his good conduct, for we can count
one year less of our three hundred years. 
But when we see a naughty or
a wicked child, we shed tears of sorrow, and for every tear a day is
added to our time of trial!'









THE END
.
