\documentclass[a4paper,11pt,onesided]{report}

\usepackage[osf=true]{libertinus}

% lang
% \usepackage[UKenglish]{babel}
% \usepackage{csquotes}

% \usepackage{subfiles}
\usepackage{verse} 

% \usepackage[style=verbose-note, natbib=true, backend=biber]{biblatex} % For referencing
% \usepackage[hidelinks]{hyperref} % Adding URLs in citations
% \usepackage{cleveref}

\usepackage{fancyhdr}
\renewcommand{\chaptername}{Thesis}

\setcounter{tocdepth}{0}
\begin{document}
\pagenumbering{roman}
\begin{titlepage}
    \begin{center}
        \vspace*{1cm}
 
        {\Huge  \textbf{Theses}}
 
        \vspace{0.5cm}
        {\textsc{Filemon Lagman}}
                   
        \vspace{1.5cm}
        
        {}

        \vspace{1.5cm}
        {}

        \vspace{0.5cm}
        {}

        \vspace{0.5cm}

        \vfill
    \end{center}
 \end{titlepage}

\tableofcontents

\pagenumbering{arabic}



% %%%%
\chapter{On the Establishment 
of a New Proletarian Revolutionary Party 
in the Philippines}


\section{}
It is impossible for a proletarian revolution 
to succeed 
without a proletarian revolutionary party 
at its forefront.

This is the definition of the Leninist concept 
of the proletarian vanguard. 
At the forefront of the workers' movement 
is a revolutionary party 
whose primary task is 
to determine the right path of advancement 
and to ensure the ultimate victory 
of the proletariat 
through the means of class struggle.

This vanguard organisation's 
class outlook is proletarian. 
Its methods are revolutionary.
And it is organised as a party of the working class.


\section{}
As the vanguard, 
what the party must attract and gather
are the most competent and the most courageous members
of the working class and other revolutionaries
who fully embrace the principles of the proletariat:
the struggle against capitalism and for socialism.

Its development as the vanguard of the proletariat 
is a long process of assembling, recruiting, and forging 
revolutionary leaders, cadres, and members.

In order to build the absolute confidence of the masses 
that this party deserves to be 
the true vanguard of the toiler and 
to build confidence in its capacity to win the struggle 
and defeat its enemies, 
this party must win---not only 
on the basis of the accuracy of its principles 
and of its loyalty in the struggle---but 
in the very knowledge, skill, and excellence 
in the science and art of the revolution. 


\section{}
This party is an organisation of class,
which views the struggle 
for progress and social justice 
from the perspective of the proletariat. 
Its stance on all matters, 
especially its program and strategy, 
is always from the point of view of the working class. 
Its theoretical guidance 
in building a consistent class line 
is the ideological system of Marxism-Leninism. 
This is the most advanced and most correct worldview 
for the revolutionary proletariat.

It is the fundamental responsibility of this party 
to practice in deep study, 
creative application, 
and continuous development 
of Marxism-Leninism to hone its vision of class. 
This party is decisive in making progressive stances 
and confident in its steps in the complex path of class struggle.

Its responsibility is to defend the accuracy of Marxism-Leninism 
and prevent various forms of misrepresentation and misinterpretation 
of its basic principles and teachings.

In the field of organisation, 
the majority of its leadership and membership
must be from the working class 
to thicken its proletarian class composition 
as a party of workers.

In the field of politics, 
its focus is the organisation 
of urban and rural workers 
and the promotion of the working class movement 
throughout the country.


\section{}
This party is revolutionary in principle and in struggle. 
It rejects all forms of reformist intent 
that corrupts the minds of the masses, 
derails the direction of their struggle, 
rots their organisation,
and fosters class treason.

As a revolutionary party in an underdeveloped country 
with underdeveloped democratic institutions, 
it has an obligation from the beginning 
to be completely secretive in nature 
and to hide all forms of struggle,
including illegal and armed.

Although, based on the existing level of class conflict,
struggle in the arena of law is paramount,
it should not confine itself to the parameter of legal struggle.

While advancing its overt movement, 
it simultaneously advances its covert movement,
which is actively revolutionary in its aims and roles. 
The fundamental duty of this party 
is to awaken and organise the toiling masses 
for armed revolution to overthrow the rotten system 
and to seize political power.

This very party 
should be the core of the people's armed forces 
as soon as the revolutionary situation ripens 
and the eve of the general revolutionary offensive arrives.


\section{}
The proletarian vanguard is not 
an ordinary revolutionary mass organisation 
but a class political party.

It is establishes itself 
not only to lead the fight for 
the comprehensive interests of one 
class---the interests of the working class,
but to also lead the struggle against other political parties,
who represent the interests of other classes 
and their reaction against the interests of the proletariat.

As a party, 
it has a defined program and strategy of struggle 
that contains the comprehensive interests 
of the proletariat 
and sets the line of advancement 
in the entire historical process of struggle.

It consists of classconscious elements of the workers' movement,
who are completely centralised in organisation. 
The centralised form of organisation 
and high level of discipline 
is a distinguishing and marked feature 
of a Leninist proletarian revolutionary party.

Only in this way can the iron discipline 
and firm command of the centre
be forged,
which are requisites for victory 
in a cruel and violent class struggle,
which is the path of advancement of the proletarian revolution.

But the centralism of a Leninist party 
is the opposite of the absolutism of a Stalinist party. 
It must be based on internal democracy 
and sharpened by a climate of ideological conflict 
in accordance with the tradition of Bolshevism.

Our concept of democratic centralism 
is Lenin's concept during his leadership of the Bolshevik party.

\section{}
At the Congress of the Establishment 
of the new proletarian revolutionary party in the Philippines, 
let us base this reorganisation 
on the long and rich experience of the communist party 
in the country and the successes and failures it has undergone,
so that the newness becomes a strength instead of a weakness.

It is necessary to correct 
its mistakes 
without rejecting the entire past,
and even if we emphasise the mistakes and crises it went through, 
the positive lessons derived from experience should not be erased.

While purifying the essence of Marxism-Leninism 
that has been bulgarised by the long history of Stalinism-Maoism 
and is now being sterilised by neo-Marxism, 
the revolutionary dedication that has been 
the hallmark of previous generations of revolutionary communists 
must be matched and surpassed.

As we shift the revolutionary movement in the country 
to the line of the proletariat and put the working class movement 
at the forefront, 
make sure that it is not separated 
from the general democratic movement. 
The class struggle must be properly integrated 
into the overall progressive movement of the Filipino people.

While aecouraging the militant party democracy away
from the suffocating absolutism and blinding fanaticism 
of the Stalinist-Maoist tradition, 
let us ensure that the anarchism and liquidationism 
of neo-Marxism do not challenge this development of democracy. 
Proletarian democracy must thicken, not dilute, the revolutionary centralism 
and iron discipline of a Bolshevik party.

\section{}
The new party's organisational structure and function
should align with its dual nature---as 
a secretive party and as 
a part of the open mass movement. 
It should also effectively incorporate 
the idea of the party 
as the vanguard for the class and 
the idea of the class as the vanguard for the people.

\section{}
Let us start forming the new party in the right step. 
On the one hand,
while advocating for open engagement in struggle, 
the party's secrecy must not be compromised.
On the other hand, 
effective leadership in the mass struggle should not be sacrificed 
for the requirements of covert action. 
The key lies in properly grasping the concept of vanguardism 
without succumbing to the pain of vanguardism.



\chapter{Class Struggle
in the Philippines}
\section{}
Our concept of revolution is completely different from that of the Zionists. 
We mainly compare it to them because the Maoist type of revolution 
is where we come from. 
There is where we have inherited the traditions of dogmatism and romanticism,
which we are correcting.

We and the Zionists both profess to embrace 
the ideological line of Marxism-Leninism. 
But their version is chronically Stalinist and Maoist. 
They don't deny it, 
but they even boast of it as the development of Marxism-Leninism. 
We, on the other hand, vomit Stalinism and Maoism 
as the vulgarisation and dogmatisation of the teachings of Marx and Lenin. 
This is the opposite of the scientific spirit and revolutionary theory 
of the great teachers of the working class all over the world.

Our difference in the concept of revolution begins 
with this difference in theory. 
This difference permeates both tactical matters 
and the study of the realities and experiences of the struggle itself. 
These differences are evident in the interpretation and application 
of the theory of class struggle which is a basic ingredient of Marxism 
and the basis of the concept of proletarian revolution.

\section{}
An essential part of our thesis 
on the current state of class conflict in the country is the analysis 
that a revolutionary situation does not yet exist in the Philippines. 
Despite intense social contradictions, 
class struggle has not yet matured 
to the historical level of revolutionary crises.

This contradicts the Zionist concept of a `permanent revolutionary situation', 
which was the theoretical basis of their strategy 
of protracted war and advance armament. 
This contradicts the Maoist concept of `chronic crisis' 
of semicolonial and semifeudal society.

These concepts do not have a theoretical basis in Marxism-Leninism 
nor in the concrete reality of the Philippines. 
These are contrary to the universal laws of class struggle. 
The armed revolution promoted on these basss 
is not a Marxist revolution 
and is far from winning proletarian goals.

\section{}
Class antagonism is what is permanent, not the revolutionary situation. 
The latter is the result of the historical development and tension of the former 
in the open and violent clash of classes 
whose central issue is in state power 
and in the form of a political crises of the reactionary system.

This historical antagonism of classes is not a direct basis 
for the immediate launch of armed struggle. 
The immediate initiation of armed revolution will be absolute in all countries 
if it were the case 
because antagonism is universal in all class societies. 
The transformation of the proletarian struggle into an armed one 
will take place when the tension of this class antagonism
reaches a high level.

The revolutionary situation does not develop independently 
of the development of the motivating forces of the revolution 
and the tension of contradictions in the reactionary camp itself.

Its material basis is the disintegration of the social condition of the classes, 
which is deeply rooted in the economic situation. 
This ignites the spontaneous resistance of the people.

The Maoist theories of `revolutionary situation' and `chronic crisis' 
are concepts of makebelieve. 
Even common sense is able to refute the blind and crazy belief 
that an economic system that is in perpetual crisis, and
because of this, is in a permanent revolutionary situation, 
can exist in the world.

It has not even a shred of connection with Marxism. 
A fiction tailored to rationally bless and sanctify the Maoist strategy. 
To the Zionists, there is no time that is not favourable for armed struggle! 
Every year `the situation has never been more favourable' for armed revolution!

For them, the revolutionary movements in countries like the Philippines 
are obliged to immediately and always be armed 
and will only advance in the principal form of guerilla warfare. 
This is a complete vulgarisation of the theory of class struggle and 
dogmatisation of its antagonistic nature.

What is impressive about Maoism is its ability to sustain 
revolutionary optimism through romanticism and fanaticism. 
What is sickening is that it distorts class conflict 
and destroys historical dynamism.

Marxism, as used by Maoism, is dogma, not a guide to action. 
Their only use of the theory of class conflict is to rationalise 
the use of revolutionary force, 
not a theoretical guide to preparing 
and launching a successful armed revolution of the toiling masses.

\section{}
The most important thing in determining the character and peculiarity 
of the strategy and tactics of the line of march of 
a Marxist-Leninist proletarian revolutionary movement 
is the objective and concrete assessment 
of the condition and class struggle in a historical situation.

Analysis of type or analysis of class must be carried out in 
a thorough and concrete manner in order to 
objectively understand and promptly calculate 
every twist and turn of the situation. 
It should not be limited to a distant `historical' perspective 
à la PSR (Philippine Society and Revolution).

Rather than the behaviour of classes being dynamic in our view, 
it becomes static in their framework. 
This is why its fails to grasp 
peculiarities of different situations.
Above all, the theory of the class struggle loses its meaning,
becomes a stump, and does not become a guide to practical action.

The importance of the PSR-style form of class analysis, 
apart from marketing, 
lies in its ability to give a 
historical perspective on the alignments of classes in society. 
This allows for a recognition of possible allies of the proletariat 
and a recognition of the stranglehold of counterrevolutionary forces. 
It is a theoretical framework for tracking the behaviour of classes, 
especially what they're for and what they're against under 
changing and evolving situations.

In the end, however, the universal theory of Marxism,
in contrast to this type of analysis, 
is already in place. 
It is not this `historical' analysis that is crucial 
to the determination of concrete proletarian policies 
in relation to other classes in changing circumstances. 
A concrete analysis of the concrete situation will determine this better.

Only the proletariat is the full and consistent revolutionary class in society. 
All other types have a reactionary character 
while others are completely reactionary in opposition to the proletariat. 
This is the reason why the analysis of types needs to be critical and dynamic.

\section{}
Class struggle is the engine of history. 
The correct interpretation and transformation of movement of society
can be done if it can be carefully studied, monitored, and managed.

It is not enough to make one's own class analysis contrary to the PSR,
for example, our conclusion that the proletariat is 
the main force of the democratic revolution, not the peasantry. 
This is not evidence that we have gotten rid 
of the catechism popularized by Mao and copied by Sison.

The meaningful use of the theory of class conflict 
is in how it can be used to explain 
concrete and historical events 
in Philippine society 
and how it can be used to discover laws and goals 
of social development based on the movement of conflicting classes.

The past three decades of the country's history,
which is also the history of the rise and fall of 
the Maoist movement in the Philippines,
must be re-examined. 
Historical events must also be explained based on the development 
of class conflict in society. 
Then,  we must thoroughly analyse our current situation 
and direction of development.

\section{}
The most important events in the past three decades that must be explained 
in the theory of class struggle are 
the imposition of the fascist dictatorship in 1972, 
the \textsc{edsa} uprising in 1986, 
and the restoration of the bourgeois-democratic form of government 
under the three successive regimes. 
Each has its own historical `introduction'. 
Martial law has a `first quarter storm'. 
\textsc{Edsa} is the `Aquino assassination', 
the series of failed coups d'êtat in the post-\textsc{edsa} regime.

But it is not our intention to make a narrative in relation 
to these parts of our history. 
We are more intent on 
discerning the fundamental explanations, conclusions, and lessons 
that are connected to the class struggle 
underlying these stages of development. 
These insights are directly relevant 
to our examination of the present situation and ongoing struggle.

The very progress and, eventually, decline of the movement 
should not stop at explanations of 
the merits or flaws of policies. 
It is important to establish 
a direct connection to class conflict 
and how it impacts the alignment of different social classes.

For instance, the boycott policy in 1986 was a serious mistake.
But is this mistake the fundamental explanation 
for why we witnessed a spontaneous uprising?
Or should the explanation delve deeper 
into the reality that the proletariat was overshadowed 
by the petite-bourgeoisie, 
which in turn became subservient 
to the interests of the haute-bourgeoisie?

There is nothing more crucial 
than recognising and identifying our mistakes. 
However, the analysis must begin 
with the context and extend to its impact on the class struggle. 
Otherwise, we will not truly learn anything. 
We will not eliminate the role of the individual 
or accident in history---the 
personal ambition of Marcos, 
the extremism of Sison, 
the assassination of Aquino, \&c. 
What is important is to explain in a Marxist manner 
why these individuals or accidents in history 
had this effect on the situation and movement of classes.


\section{}
The \textsc{edsa} revolt, 
in our basic analysis and conclusion, 
was a petite-bourgeois revolt against a fascist dictatorship. 
This is the culmination of the antidictatorship movement 
that began to erupt after the death of Ninoy Aquino in 1983. 
Its main motivating force and social support was the city's petite-bourgeoisie.

Because of this class character, 
its political content is meager and deviant. 
It was very easy to submit to the absolute leadership of the haute-bourgeoisie. 
The revolutionary character quickly faded and disappeared. 
It ended up with the simple ousting of the dictator 
and the preservation of the old system that was in danger of collapsing 
with the fall of Marcos.

In 1983, with the assassination of Aquino, 
the societal class antagonism had escalated 
to the point of a political crisis. 
The objective circumstances unfolded 
to resemble a revolutionary situation. 
However, rather than the proletariat seizing this opportunity 
to advance its class struggle 
and spearhead the antidictatorship movement, 
the opposite occurred. 
This marked the beginning of the anti-Marcos bourgeoisie 
assuming leadership and ultimately persuading 
the petite-bourgeoisie to align with their cause.

The fundamental question at hand 
is why the proletarian forces 
were not the driving force 
behind the popular uprising against the dictatorship. 
Hadn't the class antagonism between capital and labour 
intensified enough during the dictatorship 
to compel the working class 
to take the lead in the struggle for democracy? 
Furthermore, 
why did the petite-bourgeoisie emerge as the dominant force, 
albeit aligning with the anti-Marcos reactionary forces?


\section{}
We will gain no meaningful insight
if we simply attribute the mistakes 
to the boycott policy 
or the flawed tactics in the antifascist struggle. 
It is clear that the revolutionary vanguard 
of the working class 
and its effective leadership are indispensable 
for the proletarian class identity to prevail 
within the broader democratic movement of the people.

Indeed, 
even if the broad masses of the working class 
engage in spontaneous action, 
there is a risk that they might 
become absorbed and overwhelmed by 
the current of the general democratic struggle, 
leading them to lose their distinct class identity. 
They could end up merely serving 
as a subordinate and instrumental force 
for the bourgeoisie's own class interests, 
essentially becoming the tail of the bourgeoisie's agenda.

The reality of the class situation 
cannot be disentangled from 
the mistakes and deficiencies 
of the revolutionary organisation 
that assumes the role of the vanguard. 
Ultimately, 
those who must be held accountable are none other than the
the forces whose pretence as revolutionaries is class-consciousness.

In the examination of historical events,
we begin by observing the actions taken by the people of the class,
as this represents the objective reality.
Then, we seek to establish its connection
with the approach adopted by the revolutionary organisation,
which acts as a subjective factor within this reality.
Our approach involves studying first the existing reality 
and subsequently conducting a critical analysis 
of our role in the unfolding event.

\section{}

Let us immediately reject the opinion 
that the tension of class antagonism may not be sufficient for the proletariat 
to demonstrate that it is the most revolutionary class in society 
and that it is the vanguard in the struggle for democracy.

Of all the classes, 
the proletariat was the first to spontaneously, openly, and massively 
challenge and break 
the fascist terror in 1975 and 1976 
in a barrage of strikes that 
were then forbidden under the dictatorship. 
During this time, 
thousands of strikers were 
hauled away by police trucks 
to be forced into military camps.

This is how the antagonism between labour and capital intensified. 
So it is impossible not to have enough 
intensity of this antagonism in 1983--1986;
this was the period when the economic and political crisis was unfolding,
and the petite-bourgeoisie itself was massing against the fascist dictatorship.

\section{}
If the situation was tense enough, 
what, then, is the explanation as to 
why the proletariat did not met its mission and responsibility 
to intensify its own class struggle, 
insist on its own class line, 
lead the struggle for democracy, 
and absorb to its side 
the proletariat and petite-bourgeois forces 
in the city and the countryside?

Until back then, 
the level of organisation of the workers was still low, 
and in fact, it still is. 
There were only few unionised workers in the entire country,
and the politics in a bulk of them was controlled by opportunist federations.

The petite-bourgeoisie may not be so organised 
as to be considered its own class organisation,
but they exhibit different levels of organisation 
that are influenced by appeals 
that draw them into political activities. 
Moreover, 
they maintain close connections and proximity 
to their relatives within the haute-bourgeoisie class.

First is the extensive church network.
which wasactively opposed the Marcos regime during that period. 
Second are the natural organisational structures 
formed within workplaces, 
where company owners, who were already engaged 
in both overt and covert resistance 
against the dictatorship, fostered coordination. 
Third are opposition newspapers 
with significant circulation which openly attacked Marcos, 
serving as a means of organising the petite-bourgeoisie. 
Last are the various middle-force organisations 
that served as channels 
for bourgeois opposition to the Marcos regime.

In essence, 
the petite-bourgeoisie displayed effective organisation, 
not solely on its own 
but through multiple avenues facilitated 
by sections of the bourgeoisie opposing Marcos. 
This effectively pushed the petite-bourgeoisie, 
as the social base of bourgeois-liberal opposition, 
to align firmly with the antifascist cause.


\section{}
The petite-bourgeoisie outperforms the proletariat 
not only in terms of organisational capabilities 
against the dictatorship---this 
phenomenon can be attributed 
to the influence of the bourgeoisie, 
which is the most experienced class in political organisation---but 
also in terms of politicisation. 
The crucial factor here 
is the effective campaign conducted by the bourgeoisie, 
utilising democratic-liberal slogans 
that resonate with the level of political consciousness 
within the petite-bourgeoisie.

The petite-bourgeoisie is, by its nature,
inferior to the proletariat 
in political knowledge 
due to their social status. 
But what was crucial was the transformation 
of their economic disgust into political rebellion 
due to the open brutality of the dictatorship. 
Its wick is the assassination of Aquino.

It is evident that leading up to the \textsc{edsa} uprising, 
there was a noticeable absence of 
sufficient political preparation for 
the working class to emerge 
as an independent political force 
within the forefront of the antidictatorship struggle.

The Filipino workers were the first to erupt 
widely as a class against the repression 
of the Marcos regime in 1975--1976. 
But this spontaneity was not sustained as a class movement. 
Their economic struggle was not transformed 
into a political struggle 
that advocated an independent proletarian line 
opposed to the liberal bourgeois line 
and the false `democratism' of the petite-bourgeoisie.

\end{document}