\chapter{On the Establishment 
of a New Proletarian Revolutionary Party 
in the Philippines}


\section{}
It is impossible for a proletarian revolution 
to succeed 
without a proletarian revolutionary party 
at its forefront.

This is the definition of the Leninist concept 
of the proletarian vanguard. 
At the forefront of the workers' movement 
is a revolutionary party 
whose primary task is 
to determine the right path of advancement 
and to ensure the ultimate victory 
of the proletariat 
through the means of class struggle.

This vanguard organisation's 
class outlook is proletarian. 
Its methods are revolutionary.
And it is organised as a party of the working class.


\section{}
As the vanguard, 
what the party must attract and gather
are the most competent and the most courageous members
of the working class and other revolutionaries
who fully embrace the principles of the proletariat:
the struggle against capitalism and for socialism.

Its development as the vanguard of the proletariat 
is a long process of assembling, recruiting, and forging 
revolutionary leaders, cadres, and members.

In order to build the absolute confidence of the masses 
that this party deserves to be 
the true vanguard of the toiler and 
to build confidence in its capacity to win the struggle 
and defeat its enemies, 
this party must win---not only 
on the basis of the accuracy of its principles 
and of its loyalty in the struggle---but 
in the very knowledge, skill, and excellence 
in the science and art of the revolution. 


\section{}
This party is an organisation of class,
which views the struggle 
for progress and social justice 
from the perspective of the proletariat. 
Its stance on all matters, 
especially its program and strategy, 
is always from the point of view of the working class. 
Its theoretical guidance 
in building a consistent class line 
is the ideological system of Marxism-Leninism. 
This is the most advanced and most correct worldview 
for the revolutionary proletariat.

It is the fundamental responsibility of this party 
to practice in deep study, 
creative application, 
and continuous development 
of Marxism-Leninism to hone its vision of class. 
This party is decisive in making progressive stances 
and confident in its steps in the complex path of class struggle.

Its responsibility is to defend the accuracy of Marxism-Leninism 
and prevent various forms of misrepresentation and misinterpretation 
of its basic principles and teachings.

In the field of organisation, 
the majority of its leadership and membership
must be from the working class 
to thicken its proletarian class composition 
as a party of workers.

In the field of politics, 
its focus is the organisation 
of urban and rural workers 
and the promotion of the working class movement 
throughout the country.


\section{}
This party is revolutionary in principle and in struggle. 
It rejects all forms of reformist intent 
that corrupts the minds of the masses, 
derails the direction of their struggle, 
rots their organisation,
and fosters class treason.

As a revolutionary party in an underdeveloped country 
with underdeveloped democratic institutions, 
it has an obligation from the beginning 
to be completely secretive in nature 
and to hide all forms of struggle,
including illegal and armed.

Although, based on the existing level of class conflict,
struggle in the arena of law is paramount,
it should not confine itself to the parameter of legal struggle.

While advancing its overt movement, 
it simultaneously advances its covert movement,
which is actively revolutionary in its aims and roles. 
The fundamental duty of this party 
is to awaken and organise the toiling masses 
for armed revolution to overthrow the rotten system 
and to seize political power.

This very party 
should be the core of the people's armed forces 
as soon as the revolutionary situation ripens 
and the eve of the general revolutionary offensive arrives.


\section{}
The proletarian vanguard is not 
an ordinary revolutionary mass organisation 
but a class political party.

It is establishes itself 
not only to lead the fight for 
the comprehensive interests of one 
class---the interests of the working class,
but to also lead the struggle against other political parties,
who represent the interests of other classes 
and their reaction against the interests of the proletariat.

As a party, 
it has a defined program and strategy of struggle 
that contains the comprehensive interests 
of the proletariat 
and sets the line of advancement 
in the entire historical process of struggle.

It consists of classconscious elements of the workers' movement,
who are completely centralised in organisation. 
The centralised form of organisation 
and high level of discipline 
is a distinguishing and marked feature 
of a Leninist proletarian revolutionary party.

Only in this way can the iron discipline 
and firm command of the centre
be forged,
which are requisites for victory 
in a cruel and violent class struggle,
which is the path of advancement of the proletarian revolution.

But the centralism of a Leninist party 
is the opposite of the absolutism of a Stalinist party. 
It must be based on internal democracy 
and sharpened by a climate of ideological conflict 
in accordance with the tradition of Bolshevism.

Our concept of democratic centralism 
is Lenin's concept during his leadership of the Bolshevik party.

\section{}
At the Congress of the Establishment 
of the new proletarian revolutionary party in the Philippines, 
let us base this reorganisation 
on the long and rich experience of the communist party 
in the country and the successes and failures it has undergone,
so that the newness becomes a strength instead of a weakness.

It is necessary to correct 
its mistakes 
without rejecting the entire past,
and even if we emphasise the mistakes and crises it went through, 
the positive lessons derived from experience should not be erased.

While purifying the essence of Marxism-Leninism 
that has been bulgarised by the long history of Stalinism-Maoism 
and is now being sterilised by neo-Marxism, 
the revolutionary dedication that has been 
the hallmark of previous generations of revolutionary communists 
must be matched and surpassed.

As we shift the revolutionary movement in the country 
to the line of the proletariat and put the working class movement 
at the forefront, 
make sure that it is not separated 
from the general democratic movement. 
The class struggle must be properly integrated 
into the overall progressive movement of the Filipino people.

While aecouraging the militant party democracy away
from the suffocating absolutism and blinding fanaticism 
of the Stalinist-Maoist tradition, 
let us ensure that the anarchism and liquidationism 
of neo-Marxism do not challenge this development of democracy. 
Proletarian democracy must thicken, not dilute, the revolutionary centralism 
and iron discipline of a Bolshevik party.

\section{}
The new party's organisational structure and function
should align with its dual nature---as 
a secretive party and as 
a part of the open mass movement. 
It should also effectively incorporate 
the idea of the party 
as the vanguard for the class and 
the idea of the class as the vanguard for the people.

\section{}
Let us start forming the new party in the right step. 
On the one hand,
while advocating for open engagement in struggle, 
the party's secrecy must not be compromised.
On the other hand, 
effective leadership in the mass struggle should not be sacrificed 
for the requirements of covert action. 
The key lies in properly grasping the concept of vanguardism 
without succumbing to the pain of vanguardism.