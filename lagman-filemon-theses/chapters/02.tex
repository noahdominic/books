\chapter{Class Struggle
in the Philippines}
\section{}
Our concept of revolution is completely different from that of the Sisonists. 
We mainly compare ours to theirs 
because the Maoist type of revolution 
is where we come from. 
From there, we have inherited the traditions of dogmatism and romanticism,
which we are correcting.

We and the Sisonists both profess to embrace 
the ideological line of Marxism-Leninism. 
But their version is chronically Stalinist and Maoist. 
They don't deny it;
they even boast of it 
as the development of Marxism-Leninism. 
We, on the other hand, 
vomit Stalinism and Maoism 
as the vulgarisation and dogmatisation of the teachings of Marx and Lenin. 
This is the opposite of the scientific spirit and revolutionary theory 
of the great teachers of the working class all over the world.

Our difference in the concept of revolution 
begins with this difference in theory. 
This difference permeates 
both tactical matters 
and the examination of the actual conditions and experience within the struggle.
These differences are evident 
in the interpretation and application 
of the theory of class struggle,
which is a basic ingredient of Marxism 
and the basis of the concept of proletarian revolution.

\section{}
An essential part of our thesis 
on the current state of class conflict in the country 
is the analysis 
that a revolutionary situation 
does not yet exist in the Philippines. 
Despite intense social contradictions, 
class struggle has not yet matured 
to the historical level of revolutionary crises.

This contradicts the Sisonist concept of a `permanent revolutionary situation', 
which was the theoretical basis of their strategy 
of protracted war and hightened armament. 
This likewise contradicts the Maoist concept of `chronic crisis' 
of semicolonial and semifeudal society.

These concepts have a theoretical basis 
neither in Marxism-Leninism 
nor in the concrete reality of the Philippines. 
These are contrary to the universal laws of class struggle. 
Armed revolution promoted on these foundations
is not a Marxist revolution 
and is far from winning proletarian goals.

\section{}
Class antagonism is what is permanent, 
not the revolutionary situation. 
The latter is the result of historical development 
and the intensification of the former 
through open and violent clashes between classes, 
with the central issue being state power 
and the form of the political crisis of the reactionary system.

This historical antagonism of classes 
is not a direct basis 
for the immediate launch of armed struggle. 
The initiation of armed revolt 
will be immediate and absolute in all countries 
if that were true;
antagonism, after all, is universal in all class societies. 
Instead, the transformation of the proletarian struggle into an armed one 
will take place when the tension of this class antagonism
reaches a sufficiently high level.

The revolutionary situation does not arise in isolation, 
but rather emerges in conjunction with the progression of revolutionary forces 
and the intensification of contradictions within the reactionary camp itself.

Its material basis 
is the disintegration of the social condition of classes, 
which itself is deeply rooted in the economic situation. 
This is what ignites the spontaneous resistance of the people.

The Maoist theories of `revolutionary situation' and `chronic crisis' 
are concepts of makebelieve. 
Even common sense is able to refute the blind and crazy notion 
that an economic system that is in perpetual crisis and
consequently is in a permanent revolutionary situation
can exist in the world.

It has not even a shred of connection with Marxism. 
A fiction tailored to rationally bless and sanctify the Maoist strategy. 
To the Sisonists, there is no time that is not favourable for armed struggle! 
Every year `the situation has never been more favourable' for armed revolution!

For them, the revolutionary movements in countries like the Philippines 
are obliged to immediately and always be armed 
and will only advance in the principal form of guerilla warfare. 
This is a complete vulgarisation of the theory of class struggle and 
dogmatisation of its antagonistic nature.

What is impressive about Maoism is its ability to sustain 
revolutionary optimism through romanticism and fanaticism. 
What is sickening is that it distorts class conflict 
and destroys historical dynamism.

Marxism, as used by Maoism, is dogma, not a guide to action. 
Their only use of the theory of class conflict is to rationalise 
the use of revolutionary force, 
not a theoretical guide to preparing 
and launching a successful armed revolution of the toiling masses.

%4
\section{}
The most important thing in determining the character and nuance 
of strategy and tactics in the line of march of 
a Marxist-Leninist proletarian revolutionary movement 
is the objective and concrete assessment 
of the conditions and the state of class struggle in a historical situation.

Class analysis must be carried out in 
a thorough and concrete manner in order to 
objectively understand and promptly calculate 
every twist and turn of the situation. 
It should not be limited to a distant `historical' perspective 
à la PSR~(Philippine Society and Revolution).

In the framework of the PSR, 
the behaviour of classes becomes static,
as opposed to dynamic;
this is why its fails to grasp 
the peculiarities of different situations.
Above all, 
the theory of the class struggle 
loses its meaning,
becomes a stump, 
and does not become a guide to practical action.

The importance of the PSR-style form of class analysis, 
apart from popularising the movement, 
lies in its ability to give a 
historical perspective on the alignments of classes in society. 
This allows us to recognise of possible allies of the proletariat 
and recognise the stranglehold of counterrevolutionary forces. 
It is a theoretical framework 
for tracking the behaviour of classes, 
especially what they're for and what they're against 
under changing and evolving situations.

In the end, however, 
the universal theory of Marxism,
in contrast to this type of analysis, 
is already in place. 
It is not this `historical' analysis that is crucial 
to the determination of concrete proletarian policies 
in relation to other classes in changing circumstances. 
A concrete analysis of the concrete situation will determine this better.

Only the proletariat 
is the full and consistent revolutionary class in society. 
All other types have a reactionary aspect 
and others are wholly reactionary,
in opposition to the proletariat. 
This is the reason why class analysis has an obligation
to be critical and dynamic.

\section{}
Class struggle is the engine of history, and
the movement of society
can be correctly interpeted and transformed
if it is carefully studied, monitored, and managed.

It is not enough to make one's own class analysis in opposition to the PSR,
e.g.~our conclusion that the proletariat is 
the main force of the democratic revolution, not the peasantry. 
This is not evidence that we have gotten rid 
of the catechism popularised by Mao and mirrored by Sison.

The meaningful use of the theory of class conflict 
is in how it can be used to explain 
concrete and historical events 
in Philippine society 
and how it can be used to discover laws and goals 
of social development based on the movement of conflicting classes.

The past three decades of the country's history,
which is also the history of the rise and fall of 
the Maoist movement in the Philippines,
must be re-examined. 
Historical events must also be explained based on the development 
of class conflict in society. 
Then,  we must thoroughly analyse our current situation 
and direction of development.

%6 
\section{}
The most important events in the past three decades that must be explained 
in the theory of class struggle are 
the imposition of the fascist dictatorship in 1972, 
the EDSA uprising in 1986, 
and the restoration of the bourgeois-democratic form of government 
under the three successive regimes. 
Each has its own historical `introduction'. 
Martial law has a `first quarter storm'. 
EDSA is the `Aquino assassination', 
the series of failed coups d'êtat in the post-EDSA regime.

But it is not our intention to make a narrative in relation 
to these parts of our history. 
We are more intent on 
discerning the fundamental explanations, conclusions, and lessons 
that are connected to the class struggle 
underlying these stages of development. 
These insights are directly relevant 
to our examination of the present situation and ongoing struggle.

The very progress and, eventually, decline of the movement 
should not stop at explanations of 
the merits or flaws of policies. 
It is important to establish 
a direct connection to class conflict 
and how it impacts the alignment of different social classes.

For example, 
the implementation of the boycott policy in 1986 
can be considered a significant error. 
However, should this mistake be seen as the fundamental cause 
for the subsequent spontaneous uprising? 
Or should the analysis dig deeper into the underlying reality, 
where the proletariat was overshadowed by the petite bourgeoisie, 
which, in turn, 
became subservient to the interests of the haute-bourgeoisie?

Recognising and identifying our mistakes is indeed of utmost importance. 
However, it is crucial to start our analysis by considering 
the context in which those mistakes occurred 
and extend it to comprehend their impact on the class struggle. 
Without this comprehensive approach, 
we will not truly gain valuable insights and knowledge from our past errors.
We must not eliminate the role of the individual 
or accident in history---the 
personal ambition of Marcos, 
the extremism of Sison, 
the assassination of Aquino, \&c. 
What is important is to explain in a Marxist manner 
why these individuals or accidents in history 
had their particular effects on the situation and movement of classes.


\section{}
The EDSA revolt, 
in our basic analysis and conclusion, 
was a petit-bourgeois revolt against a fascist dictatorship. 
This is the culmination of the antidictatorship movement 
that began to erupt after the death of Ninoy Aquino in 1983. 
Its main motivating force and social support was the city's petite-bourgeoisie.

Because of this class character, 
its political content is meager and deviant. 
It was very easy to submit to the absolute leadership of the haute-bourgeoisie. 
The revolutionary character quickly faded and disappeared. 
It ended up with the simple ousting of the dictator 
and the preservation of the old system that was in danger of collapsing 
with the fall of Marcos.

In 1983, with the assassination of Aquino, 
the societal class antagonism had escalated 
to the point of a political crisis. 
The objective circumstances unfolded 
to resemble a revolutionary situation. 
However, rather than the proletariat seizing this opportunity 
to advance its class struggle 
and spearhead the antidictatorship movement, 
the opposite occurred. 
This marked the beginning of the anti-Marcos bourgeoisie 
assuming leadership and ultimately persuading 
the petite-bourgeoisie to align with their cause.

The fundamental question at hand 
is why the proletarian forces 
were not the driving force 
behind the popular uprising against the dictatorship. 
Hadn't the class antagonism between capital and labour 
intensified enough during the dictatorship 
to compel the working class 
to take the lead in the struggle for democracy? 
Furthermore, 
why did the petite-bourgeoisie emerge as the dominant force, 
albeit aligning with the anti-Marcos reactionary forces?


\section{}
We will gain no meaningful insight
if we simply attribute the mistakes 
to the boycott policy 
or the flawed tactics in the antifascist struggle. 
It is clear that the revolutionary vanguard 
of the working class 
and its effective leadership are indispensable 
for the proletarian class identity to prevail 
within the broader democratic movement of the people.

Indeed, 
even if the broad masses of the working class 
engage in spontaneous action, 
there is a risk that they might 
become absorbed and overwhelmed by 
the current of the general democratic struggle, 
leading them to lose their distinct class identity. 
They could end up merely serving 
as a subordinate and instrumental force 
for the bourgeoisie's own class interests, 
essentially becoming the tail of the bourgeoisie's agenda.

The reality of the class situation 
cannot be disentangled from 
the mistakes and deficiencies 
of the revolutionary organisation 
that assumes the role of the vanguard. 
Ultimately, 
those who must be held accountable are none other than the
the forces whose pretence as revolutionaries is class-consciousness.

In the examination of historical events,
we begin by observing the actions taken by the people of the class,
as this represents the objective reality.
Then, we seek to establish its connection
with the approach adopted by the revolutionary organisation,
which acts as a subjective factor within this reality.
Our approach involves studying first the existing reality 
and subsequently conducting a critical analysis 
of our role in the unfolding event.


\section{}
Let us immediately reject the opinion 
that the tension of class antagonism may not be sufficient for the proletariat 
to demonstrate that it is the most revolutionary class in society 
and that it is the vanguard in the struggle for democracy.

Of all the classes, 
the proletariat was the first to spontaneously, openly, and massively 
challenge and break 
the fascist terror in 1975 and 1976 
in a barrage of strikes that 
were then forbidden under the dictatorship. 
During this time, 
thousands of strikers were 
hauled away by police trucks 
to be forced into military camps.

This is how the antagonism between labour and capital intensified. 
So it is impossible not to have enough 
intensity of this antagonism in 1983--1986;
this was the period when the economic and political crisis was unfolding,
and the petite-bourgeoisie itself was massing against the fascist dictatorship.

\section{}
If the situation was tense enough, 
what, then, is the explanation as to 
why the proletariat did not met its mission and responsibility 
to intensify its own class struggle, 
insist on its own class line, 
lead the struggle for democracy, 
and absorb to its side 
the proletariat and petit-bourgeois forces 
in the city and the countryside?

Until back then, 
the level of organisation of the workers was still low, 
and in fact, it still is. 
There were only few unionised workers in the entire country,
and the politics in a bulk of them was controlled by opportunist federations.

The petite-bourgeoisie may not be so organised 
as to be considered its own class organisation,
but they exhibit different levels of organisation 
that are influenced by appeals 
that draw them into political activities. 
Moreover, 
they maintain close connections and proximity 
to their relatives within the haute-bourgeoisie class.

First is the extensive church network,
which was actively opposed the Marcos regime during that period. 
Second are the natural organisational structures 
formed within workplaces, 
where company owners, who were already engaged 
in both overt and covert resistance 
against the dictatorship, fostered coordination. 
Third are opposition newspapers 
with significant circulation which openly attacked Marcos, 
serving as a means of organising the petite-bourgeoisie. 
Last are the various middle-force organisations 
that served as channels 
for bourgeois opposition to the Marcos regime.

In essence, 
the petite-bourgeoisie displayed effective organisation, 
not solely on its own 
but through multiple avenues facilitated 
by sections of the bourgeoisie opposing Marcos. 
This effectively pushed the petite-bourgeoisie, 
as the social base of bourgeois-liberal opposition, 
to align firmly with the antifascist cause.


\section{}
The petite-bourgeoisie outperforms the proletariat 
not only in terms of organisational capabilities 
against the dictatorship---this 
phenomenon can be attributed 
to the influence of the bourgeoisie, 
which is the most experienced class in political organisation---but 
also in terms of politicisation. 
The crucial factor here 
is the effective campaign conducted by the bourgeoisie, 
utilising democratic-liberal slogans 
that resonate with the level of political consciousness 
within the petite-bourgeoisie.

The petite-bourgeoisie is, by its nature,
inferior to the proletariat 
in political knowledge 
due to their social status. 
But what was crucial was the transformation 
of their economic disgust into political rebellion 
due to the open brutality of the dictatorship. 
Its wick is the assassination of Aquino.

It is evident that leading up to the EDSA uprising, 
there was a noticeable absence of 
sufficient political preparation for 
the working class to emerge 
as an independent political force 
within the forefront of the antidictatorship struggle.

The Filipino workers were the first to erupt 
widely as a class against the repression 
of the Marcos regime in 1975--1976. 
But this spontaneity was not sustained as a class movement. 
Their economic struggle was not transformed 
into a political struggle 
that advocated an independent proletarian line 
opposed to the liberal bourgeois line 
and the false `democratism' of the petite-bourgeoisie.


\section{}
Fourteen years of fascist dictatorship 
created conditions that were favourable not only 
for the growth of the working class movement in the Philippines 
but also for the strengthening of the proletarian character 
and composition of the general democratic movement in the country.

There is no sufficient objective reason 
for the petite-bourgeoisie or the liberal bourgeoisie 
to overcome the level of organisation and politicisation of the proletariat 
in the long struggle against the dictatorship 
or to even become an active part 
class movement of the working class 
in the curcial moments of the regime's collapse.

If the antagonism within the reactionary camp 
escalated to a level that
pushed the liberal bourgeoisie against the fascist Marcos faction, 
if the antagonism between the fascist dictatorship 
and the middle classes in society 
escalated to a high enough level 
to push the urban petite-bourgeoisie 
in open rebellion against the regime, 
then there is no reason to say that the tension of antagonism 
between the working class and the fascist dictatorship was not enough, 
especially since at the bottom and essence of it 
is the basic antagonism between capital and labour.


\section{}
This antagonism between labour and capital, 
which was further exacerbated by fascism, 
after all, 
provides the proletariat with 
a social and political advantage
above all other classes
because it grounds the basis and purpose of its resistance 
to the fascist dictatorship 
it something deeper and more fundamental.

However, rather than serving as an advantage,
forming the basis of the proletarian movement, 
and being the basis of the proletarian movement, 
this antagonism between labour and capital 
remained confined within the economic sphere,
limited to conflicts within their respective companies.
It failed to translate into open political antagonism against the fascist regime.

So as the polarisation between the bourgeoisie 
and the petite-bourgeoisie intensified 
and the reformist opposition marched on the road against the dictatorship, 
the working class was left to fight against each other at the factory level. 
Even at the economic level, 
the working class movement, 
in these fourteen years of the fascist regime, 
did not reach the level of the general economic strike.


\section{}
There is no question 
that thousands to politicised workers 
participated in political action 
against the dictatorship from 1983--1986.
There is no doubt
that they are the most persistent in the struggle.
In fact, 
long before the bourgeoisie and the petite-bourgeoisie 
began to mobilise against the dictatorship on a large scale,
thousands of conscious workers had long been mobilised
to advance the struggle for `national democracy'.

The problem is that 
it is almost and always limited 
to the organising forces of the revolutionary movement among the workers.
Even in 1983--1986, 
with the intensity of the spontaneity of the people against the dictatorship,
this spontaneism that was demonstrated by the petit-bourgeois
was not seen within the labour sector.

The organisation of unions was lubricated,
the mobilisation of workers
into political actions hastened,
revolutionary activity intensified,
but class-wide spontaneity did not ignite. 
Spontaneous politicisation of the class 
had no overt expression.

It must have been impossible for this politicisation 
not to spread among the workers
as it happened to the entire population 
since the assassination of Aquino. 
Before this, we can be certain
that antidictatorship sentiment 
would spread sooner 
among the workers
due to union supession and economic difficulties.

The point is that it did not ignite and take off 
like the gusts of strike in 1975--1977 
as a spontaneous movement even in the critical period of 1983--1986. 
After fourteen years, 
the social antagonism and class polarization intensified,
until it reached a peak and erupted 
in the form of the EDSA uprising. 
But the Filipino proletariat could not show its revolutionary spirit. 
This is the fundamental reality of the class struggle 
that the study must explain and illustrate.


\section{}
This failure 
is the result 
of the absence 
of a true proletarian revolutionary vanguard
at the forefront of Filipino workers.
A proletarian revolutionary party whose fundamental duty should have been
to organise the class struggle of the Filipino proletariat, 
to pave the way for this class struggle, 
to clearly draw the independent line of the revolutionary proletariat, 
to win over to its side the peasantry and all poor masses, 
curb the reactionary influence of the bourgeoisie 
and false democracy of the petite-bourgeoisie, 
and ensure that the working class 
is at the forefront of the broad masses of the people 
in the democratic struggle against the fascist dictatorship.

The fundamental deficiency of the revolutionary movement 
is not yet the boycottism, 
the guerrillaism, 
or the political tactics 
throughout the period of the antidictatorship struggle. 
Its gravest sin 
is the neglect as a vanguard 
to organise the working class 
and prepare it for a decisive struggle 
rooted in the abandonment of the independent class line of the proletariat
and addiction to petit-bourgeois radicalism and romanticism.

In theory,
the Filipino worker has a revolutionary party.
In practice,
this Sisonist party 
did not act as a party of the proletariat,
and especially,
as a revolutionary vanguard of the proletarian struggle.


\section{}
If one studies the dogma and line of the Maoist party---its program, 
its strategy and tactics, 
its emphasis on organisation, 
its slogans and declarations---it 
is very easy to prove that Sison 
did not build a proletarian party 
but a party of petit-bourgeois revolutionism 
whose obsession is to revolt the peasantry and the entire people, 
not even primarily, the working class.

This party appointed itself the proxy of the working class. 
The working-class revolution itself is also given a proxy. 
It is nothing but the `agrarian revolution' 
of the guerrilla peasants 
and not even of the peasant class itself. 
Sison is satisfied that his party represents 
the `class interest' of the proletariat, 
so he no longer sees the need to lift 
the working class and its class movement 
so that it itself can lead the people's struggle 
and pull the various classes, 
especially the peasant class, 
on the side of the proletariat.

Sison thought that the party acting as a vanguard
has the same meaning and effect 
as the proletariat itself acting as 
a vanguard of the struggle for democracy. 
Until now, 
the Maoist party still does not understand 
the fundamental difference between these two concepts, 
their different effectiveness, 
and has not learned from the failure of the proletariat 
to stamp its class mark 
and lead the antidictatorship movement. 
It does not understand that 
an essential logic of building a vanguard 
is for the proletariat 
to stand as the vanguard class of the people.

Sison seems ignorant of Marx and Engels' basic declaration 
in the \textit{Communist Manifesto}:
The one who will liberate the working class is the worker himself.
But in the kind of revolution that Sison is promoting, 
it is not the class, 
not the people of the class 
that will liberate the workers 
but the peasant masses and his Maoist party.

With this kind of concept, 
let us not wonder why the Filipino workers 
were not organised and prepared 
when the revolutionary situation ignited in 1983--1986. 
The proletariat was left in the factory---and 
its `class' party on the sidelines when EDSA took place. 
The excuse that the working class did not dominate nor make its mark 
during that period 
is complete and utter nonsense.
If the middle class had enough reason to take full action, 
should the working masses be left lacking in this rebellious situation?


\section{}
What should be said is that
time has abandoned and opportuny has overtaken
the proletariat.
What there came a time for a decisive battle, 
it was not prepared,
meanwhile the bourgeoisie was `dressed' for the occasion.
They were not only ready to over throw the tryant,
but to complete their arsenal of betrayal towards the toiling masses.
As the people revolt, 
the bourgeoisie steals the budding victory.

Far too much 
were the class antagonism 
for the working class to rise up 
and lead the struggle as a defined class 
and as an independent political force. 
What is truly lacking 
is a strong sense of class motivation and consciousness, 
rooted in the clear understanding 
of its own class interests and the path of its own struggle. 
It is the fundamental responsibility of a vanguard party 
to nurture and ignite the collective consciousness 
of the working masses.

The problem is this:
What the Maost party 
has repeatedly spewed out onto the workers
was not the independent proletarian line 
during the struggle for democracy
but the `national democracy'
of petit-bourgeois revolutionism.
It dilutes the class line 
and what it thickens is the `mass line'
for the whole nation. 
Because of this, 
the proletarian class struggle was dissoved
in the middle of a general democratic movement.

The Maoist party is blabs on and on 
about a `protracted people's war' in the countryside 
while the urban people is on the eve of revolt. 
They keep blabbing on and on about a `government coalition'
when those whom it wants to partner, 
petite-bourgeoisie and the national bourgeoisie,
are already under the skirts of Corazon Aquino and the liberal bourgeoisie.


They babble on and on about the `fundamental worker-peasant alliance'
when its only materialisation,
after almost eighteen years, 
is the `merger' of the `party of the working class' 
and the `army of the peasant class' in the mountains and rural forests,
not the actual and physical linking 
of the labour movement with the peasant movement 
in the field of political struggle against the fascist dictatorship, 
against the betrayal of the liberal bourgeoisie, 
and against the false democracy of the petite-bourgeoisie.


\section{}
If the reformist revolt of the petite-bourgeoisie 
was seized by the haute-bourgeoisie
to crush the fascist dictatorship in 1986,
it was the petite-bougeois revolutionism 
that was exploited
by the reactionary state 
to crush the Filipino people 
into a fascist regime in 1972.
In these two turning points in our history,
the petite-bourgeoisie shows its crucial oscillatory character.

The armed escalation fo the situation 
by the newly-formed Maoist movement 
before Martial Law and Marcos's personal ambitions 
to stay in power 
have coincided for the imposition of martial law in the country.
While the level of antagonism displayed by both sides of the extremism 
in Philippine society 
exceeded the actual societal tensions at the time, 
it is important to recognise 
that these actions were equally influenced 
by the prevailing social conditions 
prior to the implementation of martial law.



\section{}
Before martial law was declared 
and before the Maoist party was established, 
the disintegration of the social situation in the country 
had already been accelerating.
The Philippines back then could be described as sitting on a volcano.
It is evident that 
the economic progress of Asia had already overtaken the Philippines' own,
while at the same time the initial impacts of the recession in the United States
were already being felt.

During this period, 
spontaneous working-class and peasant movements, 
as well as patriotic and revolutionary movements, 
were beginning to re-emerge.
This was after the collapse of the 50s 
and its aftermath in the following decade 
in the face of severe anticommunist repression.
Due to the defeat of the movement 
under the leadership of the old Lava party, 
the organisation and influence of the Left 
in the struggle of workers and peasants collapsed.
During the years of the resurgence of these class movements---before 
the party's reorganisation in 1968 under Sison's leadership---the 
organisation and influence of workers' and peasants' struggles 
was in the hands of reformist organisations 
such as the Federation of Free Workers 
and Federation of Free Farmers.

Organisations such as these 
were tolerated and encouraged by reactionaries 
as an alternative to organisations 
influenced by the communist movement. 
Meanwhile, Sison's influence is more concentrated 
in universities and nationalism intellectuals and personalities. 
When he split from the old party, 
he converted almost none of the peasant and labor organisations. 
He was more able to tear a piece from the PLA,\footnote{
    People's Liberation Army, also known as the Hukbalahap.
    Not to be confused with its eponym, 
    the armed wing of the Communist Party of China.
}
and this is in actuality his target 
because of his war strategy and war mongering.


\section{}
After the formation of the Maoist party in December 1968, 
its immediate priorities for organising 
were the universities in the city
while in the countryside, 
it was in the immediate formation 
of the revolutionary army 
that was also immediately established in March 1969. 
Sison's emphasis on the student movement 
was intended to bolster the propaganda movement 
for his `national democratic revolution'.
And he knows that this is the easier wellspring of party and army cadres. 
As such, Sison built a workers' party without organising the working class
and a peasant army without organising the peasant class. 
The bulk of Sison's initial forces 
were revolutionaries from the petite-bourgeoisie.

So before the outbreak of the `First Quarter Storm' (FQS) 
the situation of the class movement of urban workers and rural peasants 
was just beginning to rise and strengthen 
and was overall more under the influence of the reformists. 
Because of this, 
the articulation of the ongoing disintegration of the old order
was not yet in the form 
of widespread and militant struggles of the toiling masses.
The most prominent expression 
of the overall social situation 
and a direct attack on the reactionary system 
is the action of the young students and progressive intellectuals. 
The eruption of the widespread youth protest movement 
in the form of the FQS in the 70s 
was rooted in the objective social conditions of that time. 
However, the social force driving this eruption 
was primarily the urban petite-bourgeoisie, 
and its militant expression was the student movement. 
In this, it is not unique;
in the experience of many countries, 
the initial surge of the revolutionary movement 
takes place in this manner. 
The abnormality lies in its immediate shift 
towards armed struggle 
before it can even take root among the working masses 
and before it can advance to the level of political struggle.


\section{}
The '70s were opened by the FQS. 
After the violent repression of the first two classes, 
youth activism surged in universities and on the streets. 
The radicalisation of the spontaneous student movement began, 
departing from its previous moderate stance. 
The newly-established Maoist party 
then quickly seized leadership in the protest movement.

The reformist call for a `Non-Partisan Constitutional Convention' 
was overshadowed by the slogan,
`Overthrow imperialism, feudalism, and bureaucratic capitalism'. 
This activism quickly spread among 
youth in communities and certain urban centers 
throughout the archipelago.
The FQS occurred at a time 
when student activism throughout the world was surging 
against US aggression in Vietnam 
and in `rebellion' against various manifestations of systemic corruption. 
This was the generation of youth 
that witnessed the beginning of the US recession 
after a long period of prosperity. 
They grew up witnessing armed movements 
erupting in many backward countries 
and the raging war in Vietnam, 
which became a symbol of imperialism.

\section{}
As a result of the FQS, 
the clandestine organisation of the Maoist party rapidly expanded, 
coinciding with the initiation of armed struggle in certain regions 
and its establishment in various parts of the country. 
Key roles in building and expanding the party and armed forces 
were played by activist youth 
inspired by the Chinese revolution, 
the people's war in Vietnam, 
and the FQS.

However, 
instead of immediately linking and concentrating 
the growing number of revolutionary youth 
towards advancing the workers' movement, 
the inspiration became focused on `going to the countryside' 
and participating in armed struggle.

The Maoist party artificially intensified the situation 
through guerrilla warfare in rural areas. 
When the writ of habeas corpus was suspended 
from 1971 until the declaration of martial law in 1972,
the organisation of the Maoist party among workers and peasants 
had only just begun. 
However, armed conflicts were already escalating 
in various areas in Luzon 
while the armed forces were being established 
in various parts of the archipelago. 
The failed attempt to bring in an arms ship from China in 1971 
was a sign of the Maoist party's determination 
to intensify armed struggle, 
even with a thinning mass base. 
This was an application of Sison's conspiratorial concept of revolution. 
He launched a war without considering 
the level of class conflict, 
deliberately escalating the situation in the country 
to the level of armed confrontation 
for his Maoist people's war script.

Marcos fully embraced it. 
Marcos used the Plaza Miranda bombing in 1971 
and the exposure of the MV Karagatan in 1972 
to portray the armed rebellion and communist subversion in the country 
as a severe threat. 
He portrayed it as not just being influenced by external forces 
but also directly involving his political rivals (Aquino, Lopez, \&c.). 
He presented a conspiracy involving both the Left and the Right 
as a security threat to the nation to justify the imposition of martial law.


\section{}
The escalated and artificially intensified situation 
served the interests of the reactionaries 
when the declaration of martial law was justified, 
allowing for early and violent suppression of the movement 
while it was still emerging.

In the context of the US, 
the general policy at that time 
was to encourage the establishment of puppet dictatorships. 
This ensured the worsening crisis of the US and the global capitalist system, 
and the potential success of revolutionary movements in Indochina, 
which posed a dangerous overflow into Southeast Asia 
and became an inspiration for guerrilla movements worldwide.

For the Marcos regime, 
it responded to the interest of prolonging its rule. 
For the reactionary classes in society, 
they were more comfortable with an authoritarian state 
as long as it guaranteed the stability of their economic interests 
and if it was necessary to defend the ruling system.


\section{}
From a Marxist point of view, 
the premature and artificial intensification 
of the revolutionary forces' actions, 
not in accordance with the level and requisites of class struggle, 
was indeed a mistake. 
However, on the other hand, 
Marcos was not simply compelled by Sison's provocation. 
The Maoist rebellion was just a convenient reason for him. 
Therefore, 
in the imposition of martial law and the establishment of dictatorship, 
in line with the interests of the ruling system and the Marcos regime, 
it was inevitable for them to exacerbate 
the social antagonisms in the country 
that unfolded during the fourteen years of fascist rule.

The adventurist provocation committed by Sison was wrong. 
When martial law was declared, 
the previously surging mass movement subsided in an instant. 
In urban areas, 
the accumulated forces and machinery were almost dismantled, 
while in the countryside, 
the nascent guerrilla bases were disrupted. 
If the Moro war in Mindanao did not erupt, 
the Marcos dictatorship could have concentrated 
its entire military force on the NPA\footnote{The New People's Army} 
and crushed it completely.


\section{}
Due to the intense war in Mindanao, 
where the conditions were ripe for armed conflict, 
the Maoist party was able to recover from its initial setback. 
It gradually regrouped its guerrilla warfare 
while facing the military's attempts to crush its small units.
The military resorted to terrifying acts of fascist abuse and harassment 
against the peasant masses.

Marcos's land reform program failed, 
resulting in no progress and no social justice 
in the vast rural areas of the country. 
This created fertile ground for the emergence of guerrilla warfare. 
It started gaining strength in 1976, and by the 1980s, 
it had spread throughout the archipelago, 
with increased frequency and intensity of tactical offensives.

However, 
what fanned the movement was the antifascist sentiment of the people 
rather than an antifeudal peasant movement. 
As the entire countryside became militarised, 
there was no opportunity 
for the development of a clear and organised peasant movement 
throughout the duration of the fascist dictatorship. 
Instead of a direct intensification of the class antagonism 
between the peasant class and the landlord class, 
there emerged `armed representatives' on both sides, 
and their fight took precedence.


\section{}
Indeed, 
the Maoist guerrilla warfare spread and intensified 
throughout the country 
during the fourteen years of the fascist dictatorship. 
It aligned itself 
with the violent level of social antagonism 
imposed by martial law. 
It was only right for armed struggle 
to respond to the blatant armed oppression 
of the reactionary state.

However,
despite the advancements made in the battlefield of arms, 
the Maoist movement failed to accumulate sufficient armed strength 
to overthrow the enemy's armed forces, 
topple the reactionary state, 
and seize political power in the country.

Until the dawn of the EDSA Revolution, 
it had a long way to go 
in terms of its strength and capacity 
to reach the levels of `regular mobile warfare' and `war of annihilation', 
which are key strategies in Mao's theory of `protracted war'
to surpass the stage of strategic defence 
and advance to the stage of strategic offence. 
Until the Marcos regime fell and the revolutionary movement subsided, 
the highest level reached by the armed struggle 
was surpassing the `early substage' of the strategic defence 
and entering the `advanced substage' 
of the same strategic phase of defensive warfare and guerrilla warfare.

From a historical perspective, 
the most significant contribution 
of the Maoist armed struggle 
was its role in the downfall of the fascist Marcos dictatorship 
but not in overthrowing the entire reactionary state and system. 
In the hands of the liberal bourgeoisie, 
it served as evidence to convince the US government to abandon Marcos. 
It did not benefit the system 
and did not solve the original problem 
but rather exacerbated the issue of `rebellion' 
that was used as a justification for imposing the dictatorship.

Thus is the folly of history.
The `Maoist rebellion' that the Marcos faction used 
as a scapegoat to hold onto power 
was also employed by the Aquino faction 
to convince their imperialist patrons 
to abandon such a regime. 
Painful as it may be to admit, 
the bourgeoisie toyed around with Sison's petit-bourgeois revolutionism 
for their own factional squabbles 
and to provide a relief valve for reactionary sovereignty. 
This painful truth lies in the fact 
that Sison's waging of war 
came at the cost of the lives 
of thousands of revolutionaries and ordinary people. 
It would have been better if, after Marcos's fall, 
the movement had continued to advance. 
But what occurred instead was 
the beginning of the decline 
of the Maoist movement in the country 
and the gradual waning of revolutionary struggle.


\section{}
Sison has blamed whoever he has blamed,
but the blame should be directed towards Utrecht.
It was not wrong to launch armed struggle 
in the form of guerrilla warfare in the countryside 
during the fascist dictatorship. 
While the conditions on which martial law was imposed may have been artificial, 
it was the very act of establishing a fascist dictatorship 
that intensified the class antagonism in society to a violent level.

The flaw lies in the concept of the Sisonist party of revolution. 
It transformed revolution into war 
and confined it to war. 
Revolution was war, 
and war was revolution. 
Instead of letting the laws of class struggle 
propel the revolutionary movement 
in accordance with the concept of a Marxist revolution, 
the revolutionary struggle in the country 
developed its own separate laws of progression, 
detached from the laws of class struggle. 
This was the law of the `protracted people's war', 
a military strategy which the party adapted its political strategy. 
Instead of basing the progress of the struggle
on the laws of political struggle, 
it was the law of military struggle 
that determined the advancement of the movement. 
There is no trace of Marxism-Leninism 
in the Sisonist party's division ofclass struggle into three strategic stages: 
strategic defence, strategic stalemate, and strategic offence. 
However, 
these three strategic stages 
are the foundation of the Maoist revolution, 
and these laws are characterised 
by their military nature rather than being political.

Thus, after the assassination of Aquino in 1983 
and the sudden eruption of political struggle throughout the country, 
the revolutionary script of the Sisonist party was disrupted. 
Social antagonism spontaneously erupted in Philippine society, 
and cities boiled with political activism. 
However, the main force of the guerrilla army 
remained in the hinterlands and foothills of the mountains, 
lacking any capacity for a general offensive.


\section{}
What is truly tragic is that, 
after fourteen years, 
there was no strong and powerful workers' and peasants' movement in the country 
that could have served as the foundation 
for the revolutionary movement's own hegemony in the political struggle, 
not only against the dictatorial regime 
but also against the reformist movement 
of the antifascist petite-bourgeoisie and haute-bourgeoisie. 
Instead, the movement found itself simply keeping pace 
with multi-sectoral mobilisations 
that were not effectively able to distinguish our line
from the bourgeois line, 
aside from the overarching issues of imperialism and feudalism.

What happened was that our slogans lacked sharpness and impact 
because there was no dedicated and widespread revolutionary movement 
actively promoting them. 
It should have been the revolutionary forces 
that served as the material force 
to overthrow and expose 
the hollow and empty `democratic' slogans of the bourgeoisie.

In practice, our struggle devolved into an escalation of rhetoric and slogans. 
We juxtaposed the slogan `dismantle' 
against the bourgeois absurdity of slogans like `oust' or `resign'. 
We consistently attached `US' to the name `Marcos' 
as if the people would become anti-imperialist through this symbolism. 
It seemed as if forgetting to link it to the label of the Marcos dictatorship 
would be a mortal sin.

We ended up looking ridiculous with these slogans.
But the more fundamental point is not who has the more militant slogan, 
but rather, 
under what slogan are different progressive forces in society mobilising 
at the decisive moment of bringing down the fascist regime?

The events are clear: 
the broad masses of the people, 
not just the petite-bourgeoisie, 
rallied and were mobilized 
by the anti-Marcos bourgeois forces, 
and the undeniable proof of this 
is the popular uprising of EDSA. 
The revolutionary forces and the people 
dedicated tremendous effort and sacrifice 
during the fourteen years of life-and-death struggle 
against the Marcos fascist dictatorship. 
Millions of people were enlightened and organised 
throughout this entire period. 
During the height of the dictatorship's repression, 
the leading force in the resistance 
was the revolutionary sector. 
However, 
within just three years (1983--1986), 
the bourgeoisie seized leadership in the antidictatorship struggle. 
And at the crucial moment of the people's uprising, 
the revolutionary movement found itself on the sidelines, 
a revolutionary party standing idly by 
while the people revolted, 
while the bourgeoisie, 
in collaboration with imperialist forces, 
stole the victory of the antidictatorship movement, 
a victory earned by the people's hard work.


\section{}
In the narrative of the class struggle in the Philippines 
from FQS to EDSA, 
the distinct and progressive role of the proletariat 
is scarcely discernible. 
This is despite the fact that in urban and rural areas, 
in the places where they work, 
the class antagonism between capital and labour is relentless and unparalleled. 
Among the exploited classes,
the proletariat is the most organised and active 
in the struggle for their revolutionary interests.

If we look back at the earlier stages of the country's history,
the role of the proletariat was more prominent 
when the proletarian party was first established 
and after World War II,
than in the period we have been discussing. 
In tracing our history, 
it is evident that the petite-bourgeoisie in the urban areas 
played a more pronounced revolutionary role compared to the proletariat.

On one hand, 
there is a social basis for this phenomenon. 
If we look at it through the lens of the historical development of capitalism, 
the Philippines remains at the level 
of a petit-bourgeois, economically-oriented society 
characterised by small-scale commodity production. 
While the country's industries, agriculture, and commerce 
are indeed dominated by 
the capital of imperialist companies 
and haute-bourgeoisie 
and landlords, 
there is also a prevailing presence of the petit-bourgeois economic system 
in every corner of the country. 
This type of economy is always on the edge of prosperity and decline, 
which explains the fluctuating and ambivalent nature 
of the petite-bourgeoisie in political struggles.
Due to its occupied position in Philippine society 
and its relationship with other classes as an intermediate force, 
along with its large population size in the overall population of the country, 
the petite-bourgeoisie in urban and rural areas 
plays a strategic and crucial role 
in the political life of the nation 
and in the balance of power as a whole.

However, on the other hand, 
a revolutionary party that calls itself 
a party of the working class 
cannot truly be called Marxist-Leninist 
if it does not deeply understand and recognise 
the stronger and more powerful social foundations 
for the Filipino proletariat 
as a class and as an independent political force. 
It should not rely solely on the `representation' of the party, 
but rather on the actual, direct, decisive leadership, marking, 
and domination of the struggle for democracy 
as a preparation and prerequisite 
for the struggle for socialism.

The Filipino working class has not been able to demonstrate 
this objective truth 
rooted in its position in society 
and its class relations 
because the fundamental requirement for it to fulfill its role in history 
and organise its revolutionary struggle 
is a consciously proletarian revolutionary party. 
The Sisonist party completely failed 
to meet this need of the Filipino proletariat 
in the historical struggle against the Marcos fascist dictatorship 
and in the struggle of the Filipino people for democracy.


\section{}
This was not addressed 
by the Sisonist party 
because it is a party of petit-bourgeois revolutionism 
and not a party of proletarian revolution. 
We will not go into detail 
and repeat the comprehensive basis 
presented in Counter-Thesis I
as to why even with a compassionate outlook, 
this party will not pass as a Marxist-Leninist proletarian party.

We have already discussed 
how the Sisonist party 
wasted the historical opportunity of the Filipino workers 
during the fascist dictatorship. 
Regarding the study of class conflict in the country, 
the more serious offence of the Sisonist party 
is how it devoured the political development of the workers' movement 
and infected the consciousness of the advancing elements of the working class 
with petit-bourgeois revolutionism
that as evil as petit-bourgeois reformism,
if not worse.

This event goes beyond the question of the correctness 
of a line or policy related to the labour movement 
because it is a historical event 
that has lasted for a generation of struggle 
and continues to the present. 
At this level of direct class conflict, 
there is the deep influence and wide-ranging impact 
of the petite-bourgeoisie within the labour movement, 
particularly in the most advanced unions and workers in the country.

After the decisive defeat of the Lavaist party in the 1950s 
and the simultaneous collapse 
of the militant Congress of Labor Organizations (CLO), 
which was once recognised as the primary workers' organization in the country, 
various overtly opportunistic federations 
such as PTGWO and ALU, 
as well as reformist federations like FFW, 
vied for influence within the labour movement, 
while the remaining organisations
under the control of the Lavaist party 
became mired in economism.

Thus, 
before the eruption of the First Quarter Storm (FQS), 
the venomous influence of petit-bourgeois reformism 
and various forms of economism and collaborationism 
prevailed in the labour movement. 
Imperialism facilitated this penetration i
nto the next generation of organised labour 
as a response to the anticipated resurgence 
of the workers' movement after the era of anticommunist suppression.

When the FQS erupted following the formation of the Maoist party, 
the initial cadre rooted in the labour movement 
directed their struggle against reformism, economism, and collaborationism 
within the union movement. 
However, 
the problem was that they shifted from petit-bourgeois reformism 
to petit-bourgeois revolutionism. 
Since then, until the present, 
this current of petit-bourgeois revolutionism 
has continued within the politically conscious section of the working class, 
carrying with it the anti-proletarian, pro-peasant line of the Maoist party.

The Sisonist party cannot uplift the class consciousness 
and confidence of Filipino workers 
if the concept of revolution it teaches is bastardised---that 
the liberation of the working class will come 
through agrarian revolution and people's war in the countryside, 
with the proletariat playing a role as a participant 
in the overall national movement for freedom and democracy.

How can the class movement stand 
as an independent political force 
when it only intensifies the antagonism 
between capital and labour at the factory level 
but dilutes the anti-capitalist line 
and neglects the socialist line 
when it comes to political alignment, 
replacing it with a nationalist and democratic line 
it calls `national democracy'?

How can the working class dedicate itself 
to life-and-death struggle for `national democracy' 
if its platform does not address 
the fundamental wage slavery of capital 
and in fact has no real benefit 
if it is not mobilized and annexed 
in the struggle for socialism?

This is why the working class movement during the fascist dictatorship 
was not able to play its leading political role in the struggle for democracy. 
Both influences of the politicised petite-bourgeoisie 
are embedded in the organised ranks of labor. 
On the one hand, petit-bourgeois reformism, 
and on the other, petit-bourgeois revolutionism, 
which both champion the class interests and class struggle of the proletariat. 
In other words, instead of the infiltration of class tendencies 
from the proletariat to the petite-bourgeoisie, 
the reverse happened during the anti-dictatorship struggle.
This is the reason why, 
in the intensification of class conflict, 
in the turning point of history, 
it was not the proletariat that prevailed.

\section{}
If the Sisonist party truly had a proletarian character, 
it should have easily recognised the fundamental lesson of the EDSA uprising. 
However, due to its falsified Marxist-Leninist approach 
and bastardised understanding of class struggle and the role of the vanguard, 
it failed to even glimpse 
at the role of the proletariat in the uprising, 
not as individual participants 
but as an independent class-based and political force.

Thus 
it is impossible for this kind of party 
to grasp such issues;
it is not within the purview of their thinking. 
Their assessment focused 
on the role of the `party' and not the `class', 
as their concept of vanguardism is a concept of proxy. 
In their assessment of the mistake that led 
to the revolutionary movement losing its role in the uprising, 
they blamed their own foolishness that was boycottism 
but failed to see the larger folly 
rooted in the bastardised concept of revolution 
and the bastardised concept of the party.

Whether through participation or boycott, 
or whatever calls the Sisonist party made, 
it had little significance at that moment 
because the leadership of the antidictatorship movement 
was already in the hands of the bourgeoisie. 
The proletarian forces in the people's movement 
were lacking in political preparation 
and were entirely unprepared for the upcoming struggle, 
whether in armed or unarmed arenas.

Its funnily infuriating. 
An armed movement that indulged itself 
in armed struggle for sixteen years 
but when the moment of popular uprising arrived, 
they were woefully unprepared to take up arms, 
in urban and rural areas. 
In short, 
they learned nothing 
because the Sisonist party did not make any meaningful corrections 
after its resounding failure in the antidictatorship struggle. 
Thus, when they faced the new Aquino regime, 
their overall framework of revolutionary struggle remained the same.

US+Marcos simply became US+Aquino, 
dictatorship replaced by regime. 
The analysis they conducted lacked substantial changes 
in the political situation 
because there were no substantial changes 
in the nature of society and the state. 
Consequently, 
there were no substantial changes 
in their methods and characteristics of revolutionary struggle.

In their view, 
the peace talks were just a farce, 
hence the armed struggle continues and a protracted war ensues. 
The heated debates revolved around 
whether to expand or shrink the formations of guerrilla units 
in the countryside. 
Under the Aquino regime, 
there was a renewed intensification of guerrilla warfare 
planned to disrupt its stabilisation, 
and a new shipment of arms was attempted 
to be smuggled into the country for this purpose.

After the EDSA revolution, 
the disintegration of the Maoist movement's forces and bases in the country
did not immediately occur, 
although the mass movement in urban areas quickly subsided. 
Overall, 
the revolutionary forces remained intact, 
anticipating the new policies under the Aquino regime.

But because it did not recognise 
the fundamental errors in orientation or strategy, 
the movement continued with its old habits 
and adhered to the original script of the revolution. 
It failed to identify significant changes in the alignment of classes, 
treating Aquino's rise to power 
as a mere change in the form of reactionary rule 
without grasping its impact on the political situation and social antagonisms.

In the first year of Aquino's term 
under her `Provisional Revolutionary Government'~(PRG), 
she called for a Constitutional Commission~(ConCom) 
to amend Marcos's Constitution. 
She appointed, not elected, commissioners for this purpose. 
Simultaneously, 
she initiated peace talks with the National Democratic Front~(NDF) 
and the Moro National Liberation Front~(MNLF). 
She also immediately implemented the sequestration 
of what she called the ill-gotten wealth of the Marcos family and their cronies.

Major political issues were highlighted 
in Aquino's first year, 
but the revolutionary movement failed 
to launch effective political struggle on these issues. 
How can they effectively engage in political struggle, 
for example, on the issues of the PRG and ConCom,
when these were not even given due attention 
in the analyses and priorities by the central leadership
when these were matters 
directly concerning the consolidation of reactionary state power 
and the legalisation of exploitative social systems? 
The Maoist party did not even bother 
formulating fighting slogans related to the PRG and ConCom.

Even the peace talks were not utilised 
to confront the socio-economic issues 
that form the basis of armed struggle 
and put the Aquino regime on the defensive. 
Before negotiations on fundamental issues could progress, 
the `Mendiola Massacre' took place. 
The National Democratic Front (NDF) fell right into 
the provocation of the ultra-right 
and used it as a reason to withdraw from the negotiations. 
With the collapse of the peace talks, 
the Aquino regime initiated a total war against the NDF, 
employing the strategy of gradual constriction.

\section{}
(This section is intentionally left blank.)\footnote{The original source for this text
is the copy that is available on the Marxists Internet Archive~(\texttt{marxists.org}).
Their copy skips from Section 31 to Section 33, with Section 32 missing. 
Unable to find another copy, 
to check whether the online copy is indeed missing a Section 32 
that is available on the actual text
or it is an indexing error, where Section 33 is in fact Section 32 and so on, 
I will replicate this peculiarity here.}

\section{}
When Aquino launched the total war, 
within a few years, 
the new regime ruled by whom Sison called a witch and a spoiled brat 
accomplished what the monster Marcos failed to do in fourteen years. 
Guerrilla fronts of the Maoist movement 
were destroyed in large areas of the country 
and in many places.
the guerrilla forces totally eliminated. 
The revolutionary movement continued to decline, 
with a loss of membership in the party and the armed forces. 
Before the Aquino regime ended, 
it declared the strategic defeat of the communist rebellion.

Once again, 
the explanations provided by the leadership, 
such as the mistake of boycott, 
had no connection to the theory of class struggle 
and did not pass the perspective of historical materialism. 
Initially, 
the explanation used was that the enemy's strategy (gradual constriction) 
was new and the guerrilla forces did not immediately adapt to it in rural areas. 
In the end, `Reaffirm' emerged, 
and the blame was placed on violations 
of the fundamental principles of protracted people's war, 
claiming a deviation from the original script outlined by Sison in 1968.

When the Sisonist party launched 
the `Great Rectification Movement' 
in accordance with the conclusions of `Reaffirm', 
its ideological dogmatism had reached 
the point of intellectual insanity. 
This party's illness is terminal. 
It is not only blind to reality, 
but sees the truth in reverse.

They want to portray the original Maoist script of 1968 as correct, 
and they claim that the revolutionary movement 
experienced severe failures 
because it deviated from this script. 
But the truth is that they faithfully followed Sison's Maoist line, 
which is why the revolutionary movement 
became entangled and disjointed 
when the right moment arrived. 
In this delusion, 
a large part of the organisation broke away 
because they could no longer tolerate the garbage fed to them by Sison.

Instead of starting from objective reality and concrete events, 
Sison based his criticisms 
on what he perceived as deviations 
from his own ideas, 
his own concepts of revolutionism and guerrillaism. 
These concepts are seen as absolute and universal truths, 
so there is no need to carefully study reality to identify mistakes. 
Instead, it becomes a simple matter of identifying 
what he perceives as deviations from these concepts. 
This is the methodology of `Reaffirm', 
which essentially reaffirms allegiance to the altar of Maoism. 
Blinded by dogmatism, 
the Sisonist party cannot see the simple logic 
that the new guerrilla forces endured 
the intensity of the fascist attacks 
during the time of Marcos, 
while the larger and more experienced forces, 
with a broader mass base, 
seemed to crumble in the face of Aquino's rule.

This only proves that during Marcos's time, 
despite the unexperienced and resource-limited forces, 
the guerrilla warfare progressed and advanced 
due to historical conditions that allowed and nurtured it. 
However, these conditions suddenly changed under the Aquino regime, 
not because the fundamental social antagonisms were resolved, 
but because in the perception of the people, 
the reasons for armed struggle seemed less justifiable 
compared to the time of Marcos 
when taking up arms and living in the mountains was deemed reasonable.

It is also important to consider the reality 
that the political credibility of the movement 
was severely affected by the EDSA uprising 
due to its passive role. 
But above all, with the loss of popular and multisectoral movements 
like the anti-fascist struggle, 
there should have been aggressive movements representing the peasant masses 
to take their place. 
However, the bitter truth is that even these movements have waned.



\section{}
Instead of manoeuvring in the face of such circumstances,
the Sisonist party surrendered to revolutionary optimism, 
whose ideological wellspring was fanaticism, romanticism, and dogmatism. 
As a result, it was engulfed by the fire of counter-revolution, 
wielded by a regime that was yet to consolidate its power
and was questionable in its ability to govern.

In fact, 
aside from the lingering economic crisis inherited from the dictatorship, 
the reactionary state was riddled with internal antagonism 
that eventually erupted into a series of attempted coups.

Confronted with such enemies, 
the Aquino regime's desperate attempts 
to surpass the revolutionary movement only 
serve as proof that it is laden with grave errors 
and it teeters on the brink of disintegration.                    


\section{}
The rapid weakening and defeat of the Sisonist party---which 
had sustained eighteen years of progress
in the face of a state that had endured intense internal conflicts,
its own military divided and demoralised, 
and its populace just emerging from a popular uprising---are 
clear signs of bankrupt ideology and strategy.
The split that occurred in 1992 is not a factor here, 
as even before that, the very `Reaffirm' experienced 
a profound decline and weakening.

Due to the absence of a genuine proletarian revolutionary vanguard 
capable of organizing the revolutionary struggle of the working masses 
in that post-EDSA era, 
and the existence of a party espousing petty-bourgeois revolutionism, 
the political situation, where the reactionary state power was 
not yet consolidated, was not exploited. 
The intensified political struggle of the workers and peasants could not 
keep pace with the violent clashes within the new reactionary regime.

The imperialist forces that aided the Aquino faction ensured 
the escalation of their offensive against the revolutionary forces, 
not because there was a significant threat emanating from them, 
but to alleviate pressure on the new regime, 
exploit the anticommunist campaign for reactionary unification, 
and ease this problem to focus on consolidating state power.

Incidents such as the Mendiola Massacre and the killings of Lando Olalia of KMU 
and Lean Alejandro of Bayan bear the hallmark of imperialist conspirators. 
Their aim was to sabotage the peace talks 
and provoke the petit-bourgeois revolutionary faction of the Sisonist party 
to intensify the struggle against the Aquino regime, 
providing justification for a total war against the movement. 
Simultaneously, the central leadership was nearly depleted 
through a series of arrests, while psychological warfare was intensified, 
resulting in anti-infiltration campaigns.

Instead of prolonging negotiations, utilising the political arena for struggle, 
and concentrating forces on advancing the political struggle 
and the intensification of class movements 
among workers, peasants, urban poor, and rural communities, 
the Sisonist party fell into provocations, retreated from peace talks, 
plotted to escalate the war, and handed Aquino a reason to launch a total war.

Due to the success of the anticommunist suppression, 
Aquino faced the challenges and threats of coup attempts 
from the united forces of Marcos loyalists 
and the Enrile faction within the reactionary military with high morale. 
It is evident that Aquino's control was fragile in her early years, 
as a coup attempt almost succeeded, 
and the only thing that saved 
her was the support of the US government.

Aquino was able to defeat the petit-bourgeois revolutionism of the party 
and the attempts of putschist coups 
and managed to survive the critical period of transition 
because of the full support of the US government 
and other imperialist powers to her regime through economic aid. 
She had a solid base of support from the population 
because the petite-bourgeoisie 
and a large section of the bourgeoisie 
and landlord class were on her side. 
Most importantly, 
there were no strong movements representing the peasant masses 
against her reactionary rule.

As the power and influence of the revolutionary movement declined, 
the tide of reformism moved forward, 
marked by economic progress and political stability in the country 
compared to the era of the Marcos regime's collapse. 
Once again, the reactionary role of the petite-bourgeoisie 
in this stabilisation of the reactionary regime 
must be suppressed 
because there is no strong force of the proletarian class 
in society advocating for a revolutionary line and struggle, 
while the disintegration of the peasant class 
in the countryside continues due to capitalist development.


\section{}
Yhe Aquino regime dwiftly concluded the transition of reactionary rule 
not towards a process of democratisation from dictatorship, 
but backwards to the previous system of an elitist and bureaucratic order 
prior to martial law.
Upon the conclusion of her regime in 1992, 
she completed the stabilisation and consolidation of the reactionary state 
and managed to navigate the decaying system 
through political and economic crises without undertaking 
any fundamental reforms, 
except for the installation of a constitutional system 
characterised by bourgeois-landlord elitist democracy.

There is no other meaning to such stabilisation and consolidation 
of reactionary rule 
but the suppression of social antagonisms underneath it. 
The Aquino regime resolved the level of antagonism created by fascism 
but left unresolved the more fundamental antagonisms it sprang from.

After the fall of the fascist dictatorship and following several years 
wherein the reactionary state managed to overcome 
the political and economic crises it created, 
the social antagonisms in the country entered a new historical situation 
wherein the path of development and intensification differs 
from that of the period under the fascist regime.

We have allowed twenty years to pass without 
intensifying the class struggle in line with the proletarian path of struggle. 
It is impossible for history to let this go without demanding an account 
from the revolutionary movement.

Thus, when the succeeding Ramos regime came into power, 
it experienced the most severe internal crisis. 
Divisions emerged, 
and the general decline of revolutionary struggle in the country persisted. 
Ramos's victory itself was a slap to the revolutionary movement, 
as he was one of the cruellest henchmen of the Marcos dictatorship 
and a staunch puppet of imperialism. 
Simply because of his participation in the EDSA events, 
it seemed as though his fascist crimes were washed away.

\section{}
Ramos began his presidency in Malacañang 
as the leader elected by a small minority of the people. 
In the presidential election of 1992, 
various opposing political tendencies within the people's ranks became evident.

It was proven that the `Cory magic' had faded. 
Ramos won not because he was chosen by Aquino, 
but because he used the power and resources of the administration 
to manipulate his rivals.
As Miriam Defensor came close to winning, 
it meant that a significant section of voters favoured 
her anti-establishment image, 
but there wasn't much discrimination in her politics 
as she was ultra-conservative. 
With Mitra's resounding defeat, 
despite having the largest machinery and funds, 
his reputation as a traditional politician became a significant disadvantage. 
The 1992 election also showed that the public's memory of Marcos's sins 
seemed to have been erased, 
due to the strong campaign of Danding Cojuangco 
and the victories of many former Bataan soldiers 
under Marcos in various positions, including Ramos. 
But the most bitter outcome was the that of the somewhat progressive Salonga, 
who was defeated by Imelda Marcos.

We mention this to demonstrate the strong surge of reactionary politics 
that prevailed in the country after EDSA and the Aquino regime. 
There was no trace---if we base it on the elections 
of 1987, 1988, and 1992---that t
he Philippines was heading towards a historical upheaval 
similar to EDSA, which toppled the fascist dictatorship. 
This only confirms the decline of the influence of the progressive movement 
and the shift from reformism towards overtly reactionary politics.

Having emerged from the political crisis, 
the Ramos regime was able to focus on restructuring the country's economy 
in line with imperialist globalisation. 
It accelerated the liberalisation of the economy initiated by Aquino, 
and it cannot be denied that this led 
to a boost in the economy and business in the country, 
albeit artificial, shallow, and without benefits for the masses.

Towards the end of his regime, 
the illusionary progress that Ramos boasted suddenly shattered 
when the financial crisis hit the region, 
and the country's economy was severely affected.
The value of the peso plummeted against the dollar, 
interest rates skyrocketed, 
numerous businesses collapsed, 
and unemployment escalated. 
Under the Ramos regime, the newly intensified social antagonisms 
began to take shape, focusing on the contradictions between capital and labour, 
imperialism and the entire nation.


\section{}
During the Ramos era, 
there was also an effort to concentrate the attention of revolutionary activity 
on the workers' movement and establish a genuine vanguard party 
of the revolutionary proletariat. 
This was the result of a split within the Sisonist party. 
As we began to purify Marxism-Leninism and deepen our roots 
in the workers' movement, the Sisonists, on the other hand, further entrenched 
themselves in the mountains and adhered more strongly to their Maoist line.

We quickly drew the line of demarcation between us and the Sisonist party. 
We swiftly set sail and positioned ourselves within the workers' movement, 
in the arena of open mass movements. However, the problem remained 
in the formation of a genuine vanguard party. In the end, 
this led to another split within our ranks, derailing the momentum 
of our progress.

As we peeled away, another tendency of petit-bourgeois revolutionism emerged, 
which unfortunately found some elements within the proletarian ranks 
susceptible to their questionable motives and character.

This latest split only proves---essentially an act of sabotage 
devoid of principled substance and questionable in its timing---how 
deep the crisis of the revolutionary movement has become. 
It should be noted that the split within the Sisonist party 
is not the result of a long-standing ideological struggle, 
where the dissenting section consistently carried 
a different line and perspective.

The truth is, the split began with the absolutism 
of the Sisonist leadership and our refusal to accept 
the fundamental conclusions of the `Reaffirm'. 
It was only a year after the split that we were able to delineate 
and refine our theoretical and tactical differences (CounterThesis). 
Therefore, in our breakaway, the force we carried was also tainted 
by the lingering nicotine of ideas inherited from the Sisonist party, 
which we incorporated into our new refined theoretical and tactical framework. 
Due to the difficulty of simultaneously building the party and advancing 
the workers' movement, with the former lagging behind the latter, 
confusion arose, and dubious elements took advantage of the situation, 
fueling another split.


\section{}
It is evident that this breakaway was a clear act of sabotage, '
timed when our momentum was rapidly accelerating and our influence 
within the workers' movement and mass movement in the country was growing 
after the APEC campaign, along with the formation of a broad alliance of 
national unions. There were clear indications then that if we sustained 
the progress achieved throughout 1996, we would be better equipped to respond 
to the intensifying situation in the last year and a half of the Ramos regime.

As early as 1995, in our program, we had already calculated that 
the economic situation would worsen, and Ramos would attempt to extend his rule. 
It is also documented that we anticipated an escalation of attacks 
within our ranks to prevent the exploitation of the deteriorating situation 
and hinder our recovery from internal crisis.

We were not mistaken. The situation intensified in 1997, 
Ramos attempted to prolong his regime, 
and the crisis erupted towards the end of his rule. 
Our concerns about an attack proved true. 
We were hit by internal sabotage through Bloke, 
and as a result, our progress was disrupted, and the Ramos regime ended, 
similar to the Aquino regime, without us launching a strong counter-attack 
or witnessing a resurgence of revolutionary struggle.

The unfolding social antagonisms shaped by globalisation 
were not effectively addressed under the Ramos regime. 
It seems that this was reserved for the new Estrada regime, 
a regime that campaigned on a platform claiming to champion the `poor'.


\section{}
Estrada came to power twelve years after the EDSA uprising. 
This is nearly equivalent to the fourteen years Marcos held power as a dictator.
The past twelve years are sufficient evidence to debunk the ridiculous theory 
of the Sisonist party regarding a permanent revolutionary situation 
or social crisis. 
It is clear that the past twetelve years were a long period of reaction, 
the decline of the revolutionary movement, 
and the resolution of the crisis of the ruling system and state in the country.

However, alongside this, the past twelve years have also been a period 
of new development and intensification of class antagonisms in society, 
the emergence and deepening of a new crisis in the ruling system and state. 
The very ascent to power of a president like Estrada is a sign of 
an impending maturation of a historical situation favourable 
to the renewed intensification of class struggle and political struggle 
in the country.

History presents the proletariat and the revolutionary movement 
with a historic opportunity that should not be allowed to slip away. 
The lessons of the past thirty years since the founding and rejection 
of the party we established must be deeply absorbed and learned by 
the new party we are building, whose essential task is to organise 
the proletarian struggle in this historical era, to unite and fight 
for democracy and socialism.


\section{}
Estrada's victory as president is the result and will lead to the emergence 
of grave internal problems within the reactionary politics in the country 
at a crucial time when not only the country's crisis but also the crisis 
of globalisation is looming.

Within the ruling class itself, there are doubts about Estrada's qualifications 
to protect and safeguard the ruling system during this critical time of crisis. 
It is not simply a matter of formal titles or inherent intelligence, 
but of actual capacity and preparedness to face the responsibility as 
the principal guardian and defender of a bankrupt system.

But why did Estrada win and by the largest margin in the history of 
presidential elections in the Philippines at that?

Estrada won because there was no contender who could match his popularity 
against the reactionary forces that were concerned about his qualifications 
and character to be president.

These forces failed to unite to prevent Estrada from winning because each 
faction was consumed by its own ambitions and interests. Even the US failed 
to derail Estrada's candidacy and force various factions to unite against him. 
It quickly accepted Estrada's inevitable victory and promptly secured its
own interests.

The reactionary state became a victim of its own reactionary politics, 
and its indulgence resulted in the emergence of a Joseph Estrada, who may 
have had the ability to govern San Juan but not Malacañang. 
The fourteen years of the fascist dictatorship disrupted the 
reactionary politics in the country, 
and the past twelve years were not enough to develop qualified 
reactionary political leaders who could sustain their popularity.


\section{}
However, the decisive factor in Estrada's victory, 
despite his glaring weaknesses, was his gimmick of 
`Erap para sa Mahirap' (Erap for the Poor). 
This is the social explanation and implication of his electoral success. 
His strategists tapped into a popular sentiment, 
seizing a sensitive chord in the collective consciousness of the masses---%
the issue of poverty. 
Even though he was perceived as a fool, a womaniser, a gambler, and an addict, 
the masses were willing to take a chance on an Estrada administration 
as long as he fulfilled his campaign slogan of `Erap para sa Mahirap'.

The significant margin of Estrada's victory is, 
on one hand, a sign of the widespread desperation of the people 
amidst worsening poverty despite the economic progress 
boasted by the Ramos regime, 
and on the other, a sign of the trust or gamble placed on Estrada, 
which was based more on his role in movies than his political record, 
indicating a lack of political consciousness among the people, 
which is proportionate to the scarcity of quality leadership 
in the reactionary political reserve.

There is no doubt that the people, after twelve years, 
after the fall of the dictatorship, 
find it disheartening that this is all that EDSA has resulted in. 
The government remains corrupt, the system bankrupt, 
and the livelihood of the broad masses of the people is not progressing. 
However, the prevailing sentiment is not one of rebellion 
but rather a sense of disillusionment, 
as there is no realistic alternative that they can find or understand 
due to the credibility crisis of the progressive forces. 
Overall, the political consciousness of the people remains low, 
despite having gone through a popular uprising.


\section{}
The quality of political forces and parties participating 
in the political struggles in Philippine society may explain this. 
The revolutionary party immersed in dogma and sloganeering 
failed to offer and stimulate a high level of ideological challenge 
to the bourgeoisie in order to elevate their reactionary intellectual quality. 
The quality of political struggle 
is also a reflection of the quality of competition.

Even Aquino's political party started as a grouping of various politicians 
who were out of power during the time of Marcos and after EDSA. 
The LDP became a vessel for turncoats and traditional politicians 
who were no different from the KBL. 
Even the petite-bourgeoisie couldn't form a separate political party 
representing their own progressive aspirations. 
The LP or PDF, who pretend to be pro-bourgeois, 
instead of progressing, became completely useless. 
In the 1992 election, it bit the dust and appeared extremely miserable.

When Ramos came to power, LAKAS followed the same pattern as the KBL and LDP, 
and now, under Estrada, they did the same with LAMMP. 
Whoever is the ruling party, a new exodus of politicians is carried out 
to align themselves with power and suck up the flow of favours from Malacañang. 
Even within the reactionary camp, a true party system cannot be formed. 
This only reflects the quality of class struggle in the country.

The objective conditions of class antagonism in society are intense, 
but the struggle has not reached the level of open class struggle, 
and the classes cannot openly stand as classes. 
The landlord class is a powerful force in society, 
but it does not have its own independent political party. 
Its interests are simply hitched to the dominant bourgeois political party.

However, the bourgeoisie and the landed elite are comfortable in this situation. 
It is favourable to their interests to blur and dilute the struggles 
and keep them away from a class line. 
Only the proletariat and the peasantry have an interest 
in elevating the struggle to a certain level and forcing the reactionaries 
to openly fight in this manner. 
If this were to happen, the petite-bourgeoisie would be compelled 
to come out of its political shell 
and oscillate between the principal forces in conflict.

% 44
\section{}
The formation of the Sisonist party in 1968 was the creation 
of a petit-bourgeois--peasant revolutionary party that pretends to represent 
the proletariat and also claims to be a party of the entire nation, 
including the national bourgeoisie. 
Its identity crisis is a class hallmark of the petite-bourgeoisie. 
However, in terms of its program and practice, composition and concentration, 
it is a peasant party.

Nevertheless, its articulation of the revolutionary interests of the peasantry 
has been overshadowed by petit-bourgeois revolutionism 
and its immersion in the militarist line of protracted war. 
This is a party whose primary advocacy is the peasants' demand for land, 
yet it has produced hardly any substantial documents 
or comprehensive studies on agrarian issues.

It has undermined the revolutionary struggles of both 
the peasantry and the proletariat. 
Due to this pretence, the genuine proletarian party, 
which took more than two decades to expose, 
lacked its own independent revolutionary party that would organise 
its class struggle and raise its class consciousness during 
the crucial period of history, the fourteen years of fascist dictatorship.

The absence of such a Marxist-Leninist revolutionary party is a decisive factor
not only in why the proletarian role in the antidictatorship struggle 
did not take centre stage 
but also in why the revolutionary struggle in the country is feeble, 
affecting the overall level of political consciousness among the people 
and the development of class relations in society.

% 45
\section{}
In the current historical stage of the Philippines, only the working class, 
due to its position in society, 
has the ability to stimulate, strengthen, and provide direction to 
the open and comprehensive revolutionary struggle in the country. 
It is the only class in society that has a solid and consistent basis 
in its class interests to fully and vigorously fight, in a revolutionary manner, 
against the bourgeois-feudal state and system, as well as against 
imperialist exploitation and oppression of the Filipino people.

The hope for the advancement and eruption of a genuine peasant movement 
in the countryside in a progressive and revolutionary manner relies 
decisively on the strengthening of the proletarian class struggle in 
both urban and rural areas. It is this unity between the workers and peasants, 
this alliance between the proletariat and the peasantry, 
that determines the democratic character of the proletarian revolution 
currently being pursued.

On its own, the peasant class will face great challenges in launching 
its own class-based movement separate from the workers' movement 
in the current period. 
As a class of a bygone society and a class in the stage of disintegration, 
the peasantry has long been deprived of the position to advance 
its own class struggle. 
From a historical perspective, 
it is the bourgeoisie that should provide revolutionary leadership 
in the antifeudal struggle.

Indeed, the bourgeoisie has risen to power, holding the state apparatus, 
and even the landlord class has become a partner in ruling, 
playing a completely reactionary role in society. 
While it is possible for the bourgeoisie to carry out agrarian reforms, 
it would never do so in a revolutionary manner or even in the form 
of a mass movement. 
Within the ranks of the bourgeoisie, the only hope for the peasants 
lies in the radical and reformist sections of the petite-bourgeoisie 
who are determined to completely eradicate the remnants of the feudal order 
in the economy and politics.

Another significant reason is the long-standing process of disintegration 
and fragmentation of the peasant class due to the penetration 
of capitalist economic processes in rural areas. 
This disintegration has accelerated over the past three decades. 
The majority of the rural poor today are agricultural workers 
who were previously impoverished and middle peasants. 
In such a situation, 
only the proletariat can truly be relied upon by the peasants i
n their intense antifeudal struggle in a consistent and revolutionary manner. 
However, this cannot be achieved simply by rooting the proletarian cadres 
among the peasants while abandoning their own class organisation. 
The decisive factor lies in the advancement and eruption of 
the workers' movement and their efforts to promote progress and 
social justice for the peasant masses.


%46
\section{}
The antifeudal struggle is in the fundamental interest of the proletariat
because only through the complete destruction of the remnants of the old order 
can rural areas progress on the path of social progress and justice. 
Furthermore, the class struggle in the countryside will be liberated 
from the struggle for democracy towards the struggle for socialism.

The socialist revolution is a question of the readiness of the working class 
to launch it. 
This readiness is not solely determined by the efforts of the cadre 
to prepare the class for revolution. 
The readiness of the working class is more determined by the development 
of the historical, material, and spiritual conditions of the 
lives of the peasant masses. 
In rural areas, the main obstacles to the progress of the working class 
are the vestiges of the feudal system and the power of the landlord class 
over the peasant masses.

The historical struggle of the proletariat for democracy involves 
the abolition of the remnants of the feudal system that stifle class conflict, 
the attainment of political freedom for the peasant masses up to the level 
of political power, and the national independence of the people 
because imperialism obstructs the country's right to choose 
its own path of development.

This struggle for democracy serves as the political preparation of 
the working class in launching its own revolution, 
the revolution of the workers, the socialist revolution. 
Here lies our fundamental difference with the Sisonist party in terms of how 
we view the democratic struggle.

We adhere to the classical and scientific formulation 
of Marx, Engels, and Lenin regarding the necessity and importance of 
the struggle for democracy as the political preparation of the proletariat 
for socialist revolution. This means that the political readiness 
of the proletariat as a class to launch a successful socialist revolution 
can only be achieved through the path of democracy.

The Sisonist party uses the struggle for democracy not for 
the political preparation of the proletarian class for socialist revolution 
but rather for the seizure of state power by the `revolutionary party' 
in the name of the proletariat. After using the democratic struggle of the 
peasants and the masses, this party aims to seize political power through 
armed revolution and immediately launch the socialist revolution or 
transformation of society, again in the name of the proletariat. 
Their excuse for this kind of script of a `twostage revolution' 
is the invented theory of a `semifeudal and semicolonial' mode of production.

In this system, the working class is weak and a minority in society, 
which is why the real strengthening is done through its proxy, 
the Stalinist-Maoist `cadre party'. 
With this concept, it should come as no surprise that the political development 
of the proletariat in urban and rural areas lags behind despite the 
intensification of social antagonisms. The political preparation of the working 
class for its own revolution is impossible as long as this tendency dominates 
the movement. 
It should also not be surprising that the overall political consciousness of the 
people is backward and the class antagonism in society is feeble because the 
decisive factor in advancing the class struggle---the proletarian class---has 
been marginalised and neglected by its own `cadre', 
abandoning the organisation of the proletarian class struggle and getting 
immersed in armed struggle in rural areas.


\section{}
Over the past three decades, 
there has been continuous evolutionary progress in agriculture. 
The advancement of industry and commerce in urban areas has accelerated.
Regardless of the slower development of agriculture, 
the reality is that capitalist economic processes have 
their own laws of progress, 
and the only question is whether it will follow a rapid path 
of revolutionary transformation or a slower process of reformist evolution.

During this time, 
the formerly prevalent remnants of the feudal system of production 
have gradually collapsed and have been replaced by a capitalist 
and commercialised form of agriculture. 
What remains is the private ownership of vast lands by a few large landlords 
who remain the most powerful force in rural areas.

Currently, the relations of production and the class antagonisms 
in rural areas are complex. 
This cannot be encapsulated or simplified solely as a central issue of 
the lack or scarcity of land owned by the poor and middle peasants. 
The private ownership of land and agricultural means of production 
in exploitative agrarian relations is always a central issue. 
However, it does not automatically mean that the sole implication 
is the need for land redistribution to individual peasants.

The resolution to the issue of land ownership in rural areas should align 
with the standards of social progress, not just social justice, 
and with the standards of the proletarian class struggle, 
not just the agrarian struggle of the peasants.

It is a different matter if the mass movement in rural areas 
actually and spontaneously advanced and erupted. 
In such a situation, practice would be more important than program. 
The agrarian movement may not perfectly align with our revolutionary platform. 
What would matter is that it would advance the struggle to eradicate all remnants
of the feudal system and would intensify the class conflict in rural areas. 
However, the current situation is not yet precisely clear,
and there is no certainty about the direction or central demand 
that can ignite the rural areas. 
Even the full implementation of land reform~(CARP) is not sufficient to spur 
widespread spontaneous peasant revolts. 
Currently, various issues affecting the peasants are highlighted, 
such as land conversion, high prices of farm inputs, 
unfair pricing of agricultural products, lack of credit facilities, 
usury problems, inadequate infrastructure, 
the problem of calamities and government neglect, \&c.

Alongside this, the worsening problem of agricultural workers, 
who constitute the largest sector in rural areas, is evident. 
They are victims of exploitation by the landlord class, the capitalist class, 
and even by the peasant class itself, particularly the affluent and 
middle branches who employ their labour. 
Within their ranks, there is a widespread lack of employment, low wages, 
deprivation of rights, and feudal treatment within capitalist relations.

The interconnected problems of the peasant masses and agricultural workers 
in rural areas, exacerbated by economic crises and incessant calamities,
will fuel the agrarian struggle and class conflict in the countryside. 
From the proletariat's perspective, the key link lies in integrating 
the struggle of agricultural workers into the overall workers' movement. 
Through this movement in both urban and rural areas, unity can be strengthened 
in the struggle of the peasant class and promote its eruption as a distinct 
class-based movement.


\section{}
Whether or not the Estrada regime fulfils its platform of prioritising 
the agricultural sector of the economy, it will inevitably lead to the 
advancement of class-based antagonisms in the countryside. 
If implemented, we can be certain that it will result in the development of 
various class forces in rural areas, especially among agricultural workers, 
as his model of agrarian reform aligns with Danding Cojuangco's concept of 
modern, capitalist, and large-scale agricultural production.

Even now, the Estrada regime is already pushing for amendments to CARP to 
accommodate schemes such as stock distribution options, among others.

Under this Danding-like scheme of development, agrarian relations will 
intensify, and the alignment of various class forces will take shape due to 
the penetration of the capitalist system of production into the old 
agricultural system. 
Concurrently, there will be an accelerated disintegration of the peasant class 
and widespread proletarianization of rural areas.

If Estrada fails to actualize his agricultural program, it would mean 
further stagnation and decay in the countryside, especially as the country, 
Asia, and the world face worsening problems brought about by globalisation. 
If agriculture continues to be sidelined in the country's economic direction 
or if the modernization of agriculture takes on a localised character 
reminiscent of the NICs~(Newly Industrialised Countries) model under Ramos, 
it is more likely that the agrarian struggle will develop along 
a more classical path.

Three factors will be crucial in determining the direction and character of 
the struggle in rural areas. 
First, the nature of the government's agricultural 
policies under the Estrada regime. 
Second, the advancement of the workers' movement in both urban and rural areas. 
Third, the actions of other forces striving to advance the peasant movement.


\section{}
If `Erap para sa Mahirap' is what propelled Estrada to power, 
then this slogan will also be what nails him to the cross during 
this period of global crisis. 
Objective grounds exist for revolutionary optimism that the revolutionary 
movement will rise again and grow stronger under the Estrada regime. 
We cannot dismiss the strong possibility that this regime may not complete 
its six-year term or that it may emulate its idol Marcos in open fascist rule.

There is no doubt that Estrada's pro-poor stance is hypocritical. 
His slogan is a blatant gimmick. 
He will rule as a regime of the haute bourgeoisie and the landlord class. 
If we base our judgement on his first hundred days, 
it will undoubtedly be worse than the two regimes that preceded him, 
if not the worst in the history of the puppet republic.

This is not only due to the severe defects in his personal character 
and individual capacity as a president. 
It is also not solely because he coincided with a period 
of massive economic crisis and faces the imminent danger of an eruption. 
The additional condition is the gimmick he chose, 
which in the end he will regret. 
Not only did it raise the expectations of the masses for change, 
but it also sharpened the people's focus on the fundamental issue of poverty.

These three peculiar factors will converge to intensify the social antagonisms 
brewing under the backwards capitalism in the Philippines 
during this era of modern globalisation, 
which is sowing crises among the working class and the masses worldwide. 
The missing link is the decisive link of organising the proletarian struggle 
and the strong advancement of the workers' movement, 
with the new proletarian-revolutionary party at its forefront.

Upon assuming power, Estrada immediately initiated the `pro-people struggle' 
of his regime, but in reality, it was a struggle against the Filipino workers 
in the name of Lucio Tan and the forces of capital. 
He seems to be more `class-conscious' than previous presidents, 
with a more `pro-people nature', 
shamelessly surrendering to the capitalist class and acting as a true lackey 
to showcase his allegiance to capitalist interests.

Regardless of Estrada's level of mentality, 
he would not have reached the highest position in the country 
if his reactionary political understanding were not exceptional, 
if his instinctive actions did not align with the flow of reactionary forces. 
In the struggle with PAL, which became a gauge of his actions during the crisis, 
he did not hesitate to openly and shamelessly 
adopt an anti-union and anti-worker stance. 
His intention was clear: to appease the rapidly declining confidence 
of investors and businessmen in his regime. 
In his instinctive view, taking effective and resolute steps was a testament 
to his commitment to the interests and demands of capital, 
even if it meant sacrificing his populist gimmicks.

Estrada is not delivering a message but rather a guarantee to foreign 
and local investors and businessmen. 
He is ready to trample upon the rights of Filipino workers if it means 
propelling the `economic recovery', 
which is nothing more than a euphemism for preserving and accumulating capital 
and profits for the capitalist class and landlords. 
Now that he is in Malacañang and has started shedding his pretences, 
there is no stopping the Estrada regime as it throws away one by one 
its promises and pretences and fully submits to the dictates of the IMF 
and appeases imperialist interests, 
particularly those of the US government.

History is now giving the Filipino working class and people a new `gift' 
that threatens to bring further poverty and oppression to the country. 
This `gift' will spark rebellion among the people and serve as motivation 
to rise once again and advance a movement that will end the rule 
of capitalist-landlords in the Philippines and propel our country towards 
genuine democracy and socialist development.

The world is currently in a phase of history marked by the resurgence 
of the global working class due to the capitalist and imperialist scourge 
spread by globalisation. 
When imperialism emerged in the world during the era of colonial and neocolonial 
domination by imperialist capital, the question of national independence was 
of utmost importance. 
In this era of globalisation where national economies are being made global, 
the peculiar character of the situation is the sharpened and pronounced 
intensification of the antagonism between labour and capital 
within and across countries, in both urban and rural areas of countries 
like the Philippines. 
History is now presenting not only a new situation for the advancement 
of movements for national liberation but also for the more rapid attainment 
of the revolutionary liberation of the working masses.

